% !TEX root = ../main.tex

\chapter{绪论}

\section{研究背景及意义}
\subsection{低轨星座的快速发展与6G融合趋势}
近年来,全球通信网络正经历从地面系统向空天地一体化体系的深度演进。随着“新太空时代”的到来,卫星通信的角色发生了根本性转变,即从传统广播与地面移动通信的补充环节,跃升为支撑全球数字化与智能化转型的关键基础设施\cite{sun-SatDevelopment-satapp2023}。在此背景下,低轨卫星星座的兴起,使“广覆盖、低时延、高容量”的天地融合网络成为现实。以SpaceX的Starlink系统为例,其发展尤为迅猛,截至2025年部署卫星数量已超过7000颗,在轨通信能力约22 Tbps,预计到2030年将提升至208 Tbps,占全球在轨通信容量的约74\%\cite{wang-InternationalSatCom-satapp2025}。与此同时,OneWeb、Kuiper及SES-Intelsat等星座计划的推进,标志着低轨通信正从技术验证迈向体系化部署。我国亦将卫星互联网纳入“新型基础设施”战略布局,“国网”、“千帆”等星座计划稳步实施,为空天地一体化通信体系奠定了坚实的工程基础\cite{lin-Satinternet-broadcast2023}。综上所述,低轨卫星网络已越过萌芽期,进入规模化部署与全球组网的新阶段,成为未来通信体系中不可或缺的基础设施层。

第六代移动通信系统(6G)的愿景,进一步为低轨卫星通信赋予了新的战略定位与发展目标。国际电信联盟在2023年发布的6G愿景中,首次将“空天地一体化”确立为核心架构要素,旨在通过卫星网络实现“泛在智能连接”,以弥补地面网络在全球覆盖连续性与服务广度上的不足\cite{wang-6GSATIntro-radioCom2024}。IMT-2030(6G)推进组进一步明确指出,低轨卫星网络应成为6G架构的原生组成部分,通过在协议、频谱与调度机制上的深度融合,实现非地面网络与地面网络的无缝协同\cite{wang-Integrated6GSat-newcom2023}。这意味着,卫星系统不再仅是地面的补充,而是直接参与6G端到端服务质量保障与业务承载的关键实体。在此定位下,6G的典型应用场景对卫星通信提出了更高要求:增强型移动宽带追求更高通量;超可靠低时延通信需在动态链路中保持极致可靠与稳定;大规模机器通信则带来海量接入挑战。这些需求共同驱动卫星通信的目标从“实现连接”向“保障服务”跃迁,使其成为6G实现全球无缝覆盖与多场景服务的核心支撑。

\begin{figure}[!htp]
  \centering
  \includegraphics[width=0.9\textwidth]{figures/分布式智能发展趋势.pdf}
  \bicaption{LEO卫星系统在硬件平台、管控模式与服务类型下的智能化演进趋势}{Intelligent Evolution of LEO Satellite Systems Across Hardware Platform, Control Architecture, and Service Type}
  \label{fig:SatSys-evolution}
\end{figure}

为满足6G对网络性能的严苛要求,低轨卫星网络的技术发展呈现出星上处理智能化、管控架构分布式化的显著趋势,其演进趋势如图\ref{fig:SatSys-evolution}所示。一方面,星上处理与资源编排能力的增强,使卫星逐步具备对通信、计算、缓存资源进行联合调度的能力;可重构载荷与数字透明转发等技术的成熟,为星上业务识别与链路自适应优化奠定了基础\cite{fang-6GSatCompute-SpaceElec2023,zou-SatComSurvey-SpaceElec2023}。另一方面,人工智能技术的引入赋能卫星网络更强的环境感知与决策能力,使其能够依据实时信道状态与业务分布进行动态资源调整。尤为重要的是,随着星座规模不断扩大,集中式管控的局限性日益凸显,网络架构正加速向分布式协同演进,通过卫星节点间的轻量级信息交互与协同机制,实现跨节点的任务分解与资源优化\cite{wang-Integrated6GSat-newcom2023}。总而言之,低轨卫星通信正从单一的链路传输系统,演进为具备自治、协同与智能特征的新型网络体系。这一深刻变革反映了6G时代通信架构从“覆盖驱动”到“服务驱动”的根本性转变,也为实现端到端服务质量保障与全球一体化连接提供了持续的技术动力。

\subsection{基于人工智能的分布式网络管控的发展}
在高动态的大规模低轨星座场景下,网络管控体系正经历从集中式向分布式自治的深刻变革。随着星座规模持续扩大、星间链路切换日益频繁以及业务时延要求愈发严苛,传统依赖地面中心站进行统一决策的模式,因其响应滞后与信令拥塞等弊端,已难以适应在轨状态的快速变化。相应地,管控体系正向“星上自治—邻域协同—地面监控”的分层闭环结构演进,旨在赋予星上节点快速的本地感知与决策能力,而地面中心则专注于维护全局一致性并进行策略协调。人工智能技术的引入,正使网络逐步获得在部分可观测与资源受限条件下的自学习与自适应能力\cite{zhang-distributedSat-SCIS2025,Fontanesi-AI4SatCom-tutorials2025}。具体而言,通过在星上部署轻量级的感知与推理模型,卫星节点能够独立完成状态估计与调度决策,并借助邻域信息交互实现协同优化,从而显著提升管控的时效性与系统整体鲁棒性。可以认为,基于AI的分布式自治体系已成为当前智能卫星网络研究的基础框架。

AI驱动的分布式管控研究主要聚焦于三个关键环节:链路接入、接入域资源分配与骨干网传输。这三者分别对应低轨星座管控中连通性、可用性与可靠性的核心目标:链路接入保障卫星与用户间的动态连接与无缝切换,是维持网络连通性与业务连续性的基础;接入域资源分配负责星地无线链路的功率与时频资源调度,直接决定了数据服务的可用性;骨干网传输则关乎跨域路由与互联网接入,是影响最终用户体验的关键。因此,围绕这三个方向的AI与分布式协同研究,构成了当前低轨星座网络智能管控的核心。
\begin{itemize}
  \item 在链路接入与切换控制方面,研究已从基于参考信号接收功率的静态规则决策,转向以强化学习与图神经网络为核心的动态感知与决策机制。强化学习使星上节点能够在部分可观测环境下自主选择最佳切换时机与目标卫星\cite{sun-HO4Sat-Comletter2024,liu-HO4LargeSat-VTC2022}。图神经网络则通过学习多星间的拓扑关系与时序演化规律,为决策模型提供丰富的结构化特征输入。这两种技术的结合,使得AI模型能够在链路间歇性通断的复杂环境下维持连续接入,并有效降低信令开销。鉴于低轨卫星的高动态性导致用户可见时长极短(通常仅数十分钟),频繁的链路重建与切换对管控时效提出了极高要求。若采用集中式切换判决,则会因回程链路时延长、信令负载重而难以及时响应。为此,部分研究将切换控制功能下沉至星上,利用分布式强化学习与邻域协同更新机制实现本地策略优化\cite{ji-mobilityManagement-Network2021}。该机制使卫星能基于局部观测独立更新策略,并通过与邻星共享状态信息来保持全局一致性,从而显著降低控制时延与信令开销。
  \item 在接入网上下行信道资源分配方面,深度强化学习已成为实现跨星协同与在轨自学习的关键技术。它能够有效处理非凸的功率与频谱联合分配问题,近似最优地平衡系统吞吐量、干扰抑制与能量效率\cite{zhang-onlinepowerDRL-IJSCN2020}。在此基础上,联邦学习与分裂学习被引入以实现跨星资源调度策略的联合训练。各卫星基于本地观测到的信道状态更新模型参数,再通过星间参数聚合达成策略一致,从而在无需集中数据的前提下,获得接近全局最优的信道分配性能\cite{Elmahallawy-FedLEO-Press2024,razmi-groundFedLEO-comletters2022}。AI方法的深度应用,正推动资源管理范式从周期性的集中分配,转向持续的自适应优化,使系统能够在时变的信道条件与业务负载下维持稳定的性能与最优的能效。
  \item 在骨干网传输层面,低轨星座的动态拓扑导致星间链路的时延、可用带宽与可靠性持续变化,使得不同路径在端到端时延、能耗等性能指标间存在权衡\cite{deng-pressureRouteLEO-TVT2022}。为在动态环境下实现多目标优化,现有研究广泛采用强化学习及其图神经网络扩展进行路由决策。该方法将路由问题建模为多智能体部分可观测马尔可夫决策过程,通过邻域交互与策略迭代,实现对链路状态的持续感知与自适应调整,从而在多重约束下维持全局传输性能的稳定\cite{soret-distrubutedRouting-icmlcn2024}。此外,部分研究引入演化计算或启发式算法,以辅助图强化学习模型进行更高效的参数搜索与策略收敛,从而控制其在大规模网络中的计算复杂度\cite{kumar-fybrrlink-tnsm2022}。总体而言,AI驱动的路由优化正从传统的静态最短路径计算,迈向基于学习的多目标策略自主生成,使网络即使在拓扑频繁剧变的条件下,也能保障端到端的传输性能。
\end{itemize}

总体来看,AI驱动的分布式管控正推动低轨星座从“规则驱动”向“学习驱动”、从“集中决策”向“自治协同”演进。随着模型能力与在轨算力的提升,智能化管控的范畴正从传统的资源优化层,扩展至通信语义层。语义通信通过深入理解任务目标与信息内涵,使网络能够在传输前识别“哪些信息最具任务价值”,从而实现从“比特可靠传输”向“语义精准送达”的范式转变\cite{jiang-semanticSat-JSAC2025,hassan-semanticSat-globecom2024}。在此新范式下,星上智能体不仅依据信道状态进行链路与资源决策,还能结合任务语义与业务重要性执行跨层优化,从而在资源受限的条件下实现端到端性能的语义级增强。这一趋势标志着分布式AI管控正迈向更高层次的语义认知与智能协同新阶段。

\subsection{研究意义}
随着人工智能与分布式方法在低轨星座管控中的深入应用,网络在性能优化与工程落地方面面临新的挑战。当前研究面临的首要挑战在于跨层协同机制的缺失。尽管现有研究在接入层、资源调度层及网络层分别取得了显著进展,但大多数现有成果\cite{park-angularRouting-iotj2024,yun-spectrumSharing-JCN2022}以单层性能优化为孤立目标,缺乏以端到端服务质量为导向的统一规划,导致各层级优化目标相互割裂甚至冲突,使系统整体陷入局部最优状态。其次,AI方法虽然在非线性建模与策略学习方面展现出优势,但其高复杂度、可解释性不足及算力需求大等特征,使得算法在星上算力受限、链路时变及实时性要求严格的条件下难以保持鲁棒性与泛化性\cite{Guimaraes-SatComputing-tutorials2025}。与此同时,低轨星座固有的高速运动与弱中心化结构进一步放大了这些矛盾:拓扑与信道状态变化频繁,而星间通信受限,集中式决策难以及时响应,分布式学习又受到观测不完备与通信开销的制约。由此,如何在受限通信与计算条件下构建具备高感知、强协同和自适应优化能力的智能管控机制,成为实现低轨星座在强动态环境下端到端QoS保障的关键科学问题。

面向此需求,本文的研究聚焦于移动性管理、接入网资源调度与骨干网多跳传输三个关键方向,旨在系统性提升网络的感知、跨层优化能力及用户体验。在移动性管理方面,核心目标是在保障用户接入公平性的同时优化系统效率,重点攻克多目标优化的可解释强化学习、高效智能体交互及大规模分布式训练等难题,以实现切换决策的公平性、服务连续性与算法效率的统一。在接入网资源调度方面,研究致力于实现细粒度的跨层QoS保障与动态自适应能力,其核心是突破AI在环境动态建模与预测方面的瓶颈,构建具备环境自适应性的优化框架,以提升算法在非平稳信道与业务波动下的稳定性。在骨干网多跳传输方面,研究着眼于降低传输时延与重传开销,并在多重动态约束下实现网络容量与时延的联合优化,其关键在于突破分布式路由中的图结构信息感知瓶颈,并探索与语义传输的跨层协同机制,从而保障端到端传输的高可靠与高效率。

在学术层面,本研究通过融合分布式强化学习、凸优化与跨层协同机制,推动了智能管控理论在复杂动态网络中的深化与拓展。在移动性管理方向,本文提出了一个通用且可解释性强的强化学习框架,为刻画系统公平性与个体接入效用间的动态权衡提供了新的理论模型。在接入网资源调度方向,建立了一套低复杂度、具备边缘自治能力的资源联合调度框架,为动态环境下星上资源的跨层协同分配提供了可实现的优化路径。而在骨干网多跳传输方向, 研究着力于提升图结构信息聚合的效率与拓扑辨识能力,揭示了拓扑动态性对路由稳定性与语义性能的影响规律,为智能路由与语义通信的协同设计提供了方法论支持。总而言之,本研究从理论建模、算法设计到性能验证形成了完整的研究体系,系统性地揭示了分布式智能管控在多目标决策、跨层协同与动态感知中的核心机理,为低轨星座智能化管控的理论发展提供了统一的研究范式。

在工程与应用层面,本文的研究为低轨星座实现星上智能管控与边缘自治提供了关键技术基础。研究成果直接支撑了星座在动态环境下的分布式接入控制、资源自适应编排与多跳可靠传输等核心功能,显著增强了其自主业务保障能力。在体系集成方面,本研究提出的管控框架与算法设计与现有5G非地面网络标准体系相兼容,为卫星网络从集中式管控向分布式自治的平滑演进提供了具体且可验证的技术路径。同时,所形成的多智能体协同与跨层优化思路,可为未来6G空天地一体化网络中的智能资源管理、语义通信及端到端QoS保障等应用奠定方法论基础。因此,本研究不仅针对具体技术挑战提出了解决方案,更为卫星网络向智能化、自治化与体系化发展的工程实践提供了可迁移的技术支撑与系统范式。

\section{国内外研究现状及分析}
自第三代合作伙伴计划(3rd Generation Partnership Project,3GPP)在 Release 15/16 阶段启动基于新空口的非地面网络(New Radio based Non-Terrestrial Networks, NR-NTN)的系统性研究以来,已在接入流程、链路层机制、时间同步及系统架构映射等方面形成较为完整的技术框架\cite{3gpp-38811,3gpp-38821}。基于此背景,学术界围绕信道建模、随机接入、负载迁移、路由机制与拥塞控制等关键问题\cite{Niephaus-QoSProvisioning-ComSurveys2016,Radhakrishnan-PhysicalNetwork-ComSurveys2016}展开了广泛研究,已构建了传统卫星网络优化的主流技术范式。然而,这些方法多基于解析建模、规则驱动策略和分层优化框架,通常依赖于对网络状态的理想化描述以及相对稳定的系统假设。在低轨卫星高速运动、频繁星间切换和业务动态性极强的场景下,现有方法在实时响应、用户服务质量(QoS)保障及服务连续性等方面面临显著挑战。

在此背景下,面向低轨卫星星座的分布式管控机制与基于人工智能的边缘自治技术逐渐成为新的研究热点,相关成果已在接入控制、资源分配与路由决策等任务中取得初步进展。例如,已有研究采用深度强化学习实现功率与带宽的动态调度,利用图神经网络(Graph Neural Networks, GNN)进行拓扑感知的多跳路由选择,借助时序预测模型预估业务到达与切换状态,或基于Transformer构建信道预测模型。这些工作展现了数据驱动方法在高动态星座环境中的潜力,但整体仍呈现“碎片化”特征:多数研究聚焦于单一任务,缺乏对星座级跨层关联的系统性建模;同时,对算法鲁棒性、泛化能力以及星上处理复杂度等实际约束的研究仍显不足,尚未形成支持星座级智能管控的完整方法体系。需要指出的是,低轨卫星星座的端到端服务保障涉及多个紧密耦合的关键环节,包括:分布式管控架构所提供的全局协调与策略协同、接入与切换过程中的连接维持与负载均衡、网络切片对多业务资源的差异化调度,以及路由与传输机制所决定的多跳可达性与端到端时延性能。上述环节共同构成端到端数据传输的逻辑链路。因此,本节将围绕分布式管控架构、用户接入与切换、网络切片以及网络路由四个方向,系统梳理相关研究进展:一方面梳理地面移动通信系统与传统卫星网络在以上环节形成的代表性技术,分析卫星系统对地面机制的继承与适配方式及其在低轨星座场景中的适用性与局限;另一方面,综述现有基于AI与分布式算法的研究成果与尚存问题,从而为本文后续研究奠定技术基础并明确方法论路径。

\subsection{低轨卫星网络分布式管控架构}


\subsection{面向用户切换的低轨卫星移动性管理技术}
SDN 将控制平面和数据平面分离,实现了网络配置的可编程性,提供了更灵活的网络管理方式[57]。在 SDN 网络中,控制器是负责管理和配置网络设备的核心。控制器部署问题(Controller Placement Problem,CPP)包括 SDN 控制器的物理位置、数量和角色,其设计直接影响网络的性能和效率。CPP 可以视为一个单目标或多目标的优化问题,通常包括最小化网络延迟、降低维护成本、提高网络可靠性等关键目标[58,59]。

在传统网络架构中,控制器的分配通常是固定的,缺乏动态调整能力。在数据中心网络中,由于交换机被静态地分配给控制器,控制器的静态配置可能会因流量动态变化导致响应时间延长和维护成本增加。随着网络流量的波动,故障的发生以及控制器负载的不均衡,静态分配难以满足实时性与效率要求。因此,支持控制器动态分配机制对于提升网络适应性与管理效率具有重要意义。在动态控制器分配中,Tao Wang 等人[60] 通过在线优化方法来最小化响应时间和维护成本,将控制器分配问题划为一个稳定匹配问题,使用户随机地匹配到域控制框架优化控制器的分配。Samaresh Bera 等人[61] 提出了基于流量特定要求的动态控制器分配方案,通过自适应阈值选择和动态稳定匹配博弈来进行流量和控制器之间的分配,以减少流量时延和控制开销,并提高流量的 QoS 保证。Ashutosh Kumar Singh 等人[62] 提出了基于 Varma 优化的控制器部署问题求解方法,该方法旨在最小化网络的平均延迟。此外,还研究了控制器部署方案的能耗,包括部署成本和能耗,以实现绿色经济。Ilora Maity 等人[63] 提出了基于 IoT 流量的能量感知控制器部署方案,在带内控制平面下设计合理的控制器部署和路径选择来减少能量消耗。Alejandro Ruiz-Rivera 等人[64] 提出了降低控制器部署能耗算法,通过关闭尽可能多的链路来节省能量,同时考虑延迟、链路和控制器负载等约束。但是控制路径的重新路由可能会导致控制器过载。Ying Hong 等人[65] 提出了降低能量消耗的控制器部署方法,提出在时延和控制器负载的约束下最小化网络的能量消耗的二进制整数规划问题,并设计了遗传启发式算法对该问题进行求解。Adriana Fernandez-Fernandez 等人[66] 提出面向能效优化的流量工程方法,通过减少满足特定流量需求所需的链路数量,从而降低 SDN 网络中的能量消耗。其他研究还提出了多个控制器负载分布式管理,针对网络延迟、可靠性和负载平衡等多目标问题,提出了控制器部署的多维优化方法[67]。在控制器部署的优化算法方面,常采用线性规划[68]、贪心算法[69] 和模拟退火算法(Simulated Annealing Algorithm,SAA)[70] 等启发式方法,以求解不同场景下的最优控制器部署方案。

然而,以上研究多数集中在静态场景下的动态控制器部署方案。在卫星网络中,控制器部署面临更复杂的挑战。由于卫星时刻处于运动动态中,频繁地动态场景下的控制器部署策略[71]。在软件定义的卫星或天地一体化网络架构中,控制器的部署位置和数量,会影响整个网络系统的控制方式和管理策略。控制器通常部署在卫星网络和地面网络中,其选择直接影响到网络的管理效率、控制延迟和负载均衡能力。不同的控制器部署优化目标决定了总体网络的性能。此外,网络控制在大规模卫星星座中起着至关重要的作用,通过控制器协调大量网络节点,以保障未来空间通信网络的运行和服务的有效性和可靠性[72]。表 1-4 对比了软件定义卫星网络中不同控制器优化策略。

地面网络具有强大的计算能力和存储能力,地面网络与卫星网络的信息交
换即由卫星网关负责。Jiajia Liu 等人[73] 提出了针对软件定义的天地一体化网络中控制器和卫星网关部署问题的研究,探讨了卫星网络的部署问题以最小化平均延迟,并提出了基于模拟退火的近似解决方案。在控制器和网关的联合部署问题中,谋取令北区网络可靠性并满足延迟约束目标,采用基于模拟退火与聚类结合的混合算法对问题进行求解。Hua Qu 等人[74] 主要研究了在 UAV 上的控制器动态部署问题,以实现控制器的全球范围部署。Yongpeng Shi 等人[80] 讨论了天地一体化网络中的跨层网关选择问题,并将其定义为一个约束优化问题,利用贪心算法从空中网络中选取最优网关集合,作为连接地面层与卫星层的数据传输中继。Yizhou Shen 等人[81] 联合部署网关和控制器的问题,在改进的密度峰值聚类基础上,引入模拟退火算法以最小化负载平均差异率,并确定合理的控制器与网关数量。总
体上,地面控制器属于静态控制器,能耗与时延较长,难以满足动态流量需求。

由于卫星网络使用相对较低的延迟,控制器可以部署在 GEO 或 LEO 卫星上。Ariel Papa 等人[75] 考虑了用户的地理位置和时区变化对流量需求的动态影响,利用整数线性规划(Integer Linear Programming,ILP)来最小化流量延迟。LEO 卫星网络的流量分布与特定地理区域的活动密切相关[83]。例如,人口密度较高的区域往往会面临更为集中的通信需求。这种需求集中不仅导致卫星网络中相关区域的流量激增[84],而且也进一步加剧了网络资源的紧张。由于流量需求的不均匀分布,导致链路存在流量过载等情况。在卫星网络中,流量工程(Traffic Engineering,TE)是高效优化网络流量的重要手段[85]。

\subsection{低轨卫星网络的动态切片技术}
LEO 卫星网络的流量分布与特定地理区域的活动密切相关[83]。例如,人口密度较高的区域往往会面临更为集中的通信需求。这种需求集中不仅导致卫星网络中相关区域的流量激增[84],而且也进一步加剧了网络资源的紧张。由于流量需求的不均匀分布,导致链路存在流量过载等情况。在卫星网络中,流量工程(Traffic Engineering,TE)是高效优化网络流量的重要手段[85]。

卫星网络的负载均衡算法可分为全局负载均衡路由算法与局部负载均衡路由算法。Xia Deng 等人[86] 提出了限制传输时长的路由方法,通过结合卫星之间的欧几里得距离和背压路由,将每个流的传输限制在特定区域内以减少传输冗余,并利用低拥塞的路径进行动态传输,以有效平衡整个网络的流量负载。Jiang Liu 等人[87] 提出了选择性分裂负载均衡路由算法,旨在通过降低低优先级链路的使用频率实现负载均衡。为此,他们改进了选择性迭代 Dijkstra 算法,以减少节点重复率并优化路径计算过程;同时,采用选择性拆分策略,将流量从拥塞节点引导至邻近节点,有效降低了计算资源需求与信令交互量。Guanghua Song 等人[88] 提出了一种基于交通灯的智能路由策略 TLR,该方法利用一组交通灯指标记录当前节点与下一跳节点的拥塞状态。数据包在被预定路径传输的过程中,可根据中继节点的实时交通灯颜色动态调整路由路径,实现了“预先规划与实时调整”的有效结合,从而为每个数据包选择近似最优的传输路径。TLR 采用局部负载均衡的路由算法,控制信息开销较小,但由于不能有效规避拥塞区域,易导致数据包被反复转发至同一拥塞区域,从而引发数据拥塞问题。全局负载均衡可以动态调整网络资源的分配,避免某些节点或链路过载,提升整个网络的资源利用率。Tarik Taleb 等人[89] 提出了显式式负载均衡算法 ELB,用于解决非静止卫星通信系统中因用户分布不均导致的链路拥塞问题。ELB 算法在卫星之间同步拥塞状态信息,在即将发生拥塞时,卫星可以请求邻近卫星来减缓数据转发速率,并通过寻找其他较少拥塞的路径传输数据,避免了卫星拥塞和数据丢包,实现了更好的流量分配。ELB 采用分布式优化方法,但它又依赖于不同节点之间的通信,可以视为全局负载均衡的方法。Yong Lu 等人[90] 提出了分布流量平衡路由的方法,保障了路由的生存性并提供流量平衡能力,还有效减少了由于卫星故障信息泛洪带来的计算和存储开销。Chunxiao Liu 等人[91] 利用剩余负载率进行混合路由,并将其与遗传算法和蚁群算法相结合,有效地平衡了网络流量,解决了 LEO 卫星网络中的负载均衡问题。

现有卫星网络的 TE 解决方案面临着可扩展性和复杂性相关的问题。分段路由(Segment Routing, SR)是一种新型的网络路由架构,利用多段路径的方式,使得网络中的每个路由决策都可以预先定义,并通过标志堆叠传递,能够有效提升网络流量管理的灵活性和效率[92,93]。随着分段路径的增加入,合并应增
加开销的成本,但有更灵活的路径选择。段段的搜索空间较小,快速调整和重整[95] 并基于机器学习的流量预测方法[96] 等可提高复杂动态环境中的流量波动和拥塞问题,显著提升流量工程的效率和灵活性。在 LEO 卫星网络中,Wei Liu 等人[97] 提出了基于分段控制的流量调度自适应路由算法,根据网关之间的位置和反向时隙动态划分载区和重载区,在重载区的感知层节点的载载延时、负载区和重载区采用了不同的路由规则,解决了由于地面网关分布在有限区域内而引起的网络拥塞问题。为了解决卫星 Internet 网络,Menghan Wu[98,99] 等人提出在 LEO 卫星网络中优化 ISL 性能的方法。对于广播传输场景,采用了基于内容分发的路由策略;而对于多播传输场景,利用了矩形扇射树构和链路负荷的斯坦纳树约提高带宽利用率。此外,为了应对卫星网络中的意外故障,SR 的快速重路由机制可以快速响应网络问题。Shengyu Zhang 等人[100] 提出了基于分段路由的流量拆分算法,并研究了两种快速重路由机制,以保证数据包在网络中的平衡转发和快速恢复。Xun Chen 等人[101] 提出了结合集中计算与分段路由的混合快速重路由方案,卫星可以预计算重路由路径,并将路由规则分发给具备缓存发送资源的卫星。

在卫星网络中,源节点和目的节点之间通常存在多条路径。为提高资源利用率与网络灵活性,引入多路径路由技术能够有效提供更高效的传输方案[102,103]。传统的等价多路径(Equal-Cost Multi-Path,ECMP)路由[104] 根据数据包报头的某些元素进行静态流量分割以实现路径调度,但没有考虑节点和延迟对网络参数的限制,仍会发生网络拥塞。在 SDN 网络中,Yurii Kulakov 等人[105] 提出由 SDN 控制器集中生成路由信息,并结合多路径路由策略,进一步提升动态流量重配置的效率。Md. Sajid Hossen[106] 使用了链路带宽消耗作为路径权重,并通过深度优先搜索(Depth First Search,DFS)选择最佳路径,在实现多路径负载均衡的同时提高了 SDN 的 QoS 保证能力。多层卫星网络由于其广泛的覆盖范围和高网络容量,可以构建高效的通信网络。由于流量拥塞,多层卫星网络同样可能会出现吞吐下降和严重的端到端延迟等情况。Wenchao Xu 等人[107] 提出了基于网络编码的多径路由算法,通过混合卫星网络传输流量,以有效满足多种数据流量的传输需求。Xiaoylu Liang 等人[109] 提出了基于深度强化学习(Deep Reinforcement Learning,DRL)的路由优化算法框架用于大规模卫星动态多路径复用策略,提高 LEO 卫星网络中的路径发现准确性和多路径转发性能。

可以看出,现有研究围绕卫星网络的流量工程与路由优化问题,已提出多种具有针对性的算法与策略,覆盖局端与全局负载均衡、分段路由、多路径传输等多个方面,并逐步引入了基于 SDN 的创新方法,显著提升了网络的灵活性与资源利用效率。然而,面对卫星网络的高时空动态特征和拓扑不均衡的快速重组带来的复杂调度约束问题,现有方法在全局最优性、实时调度能力和跨层协同机制方面仍存在不足,仍需进一步关注大规模 LEO 卫星网络的多路径优化与路由决策机制,以实现更高效、更鲁棒的卫星网络流量管理体系。

\subsection{低轨卫星网络语义通信与路由技术}
卫星边缘缓存是提升内容传输服务质量的关键手段。在卫星网络中部署边缘计算与存储功能,可使内容更接近用户侧,既显著降低数据传输时延,又有效减轻地面网络负载[110]。每颗卫星除了充当网络路由器外,也具备类似云计算功能,使 LEO 卫星星座具备在轨处理功能,能够根据服务的提供算力。Debopam Bhattacherjee 等人[111] 分析了卫星边缘计算(Satellite Edge Computing,SEC)模式可以在内容分发、边缘计算、多人游戏等领域带来潜在的益处。Bradkey Denby 等人[112] 提出了在轨边缘计算 OEC 架构,旨在减少对地面基础设施的依赖,从而缓解传统管控架构在星座规模扩大时面临的通信瓶颈问题。由于卫星的通信与计算资源具有时变性,合理选择服务实体的部署节点并在恰当时机完成迁移,对于保障用户体验质量具有重要影响。Qing Li 等人[113] 提出了在线服务放置算法,通过结合了 Lyapunov 优化和 Gibbs 采样方法提高资源受限情况下的服务覆盖率,解决了 LEO 卫星星座中 SEC 节点在服务覆盖与鲁棒性之间的冲突问题。Ziqi Li 等人[114] 根据卫星资源的时变性与可见时间限制,动态选择最优服务节点并适时迁移服务实体,并将该问题建模为马尔可夫决策过程,并提出了基于惩罚–近端策略优化的在线部署方案 COMPOSE,旨在在延迟约束与迁移成本之间实现平衡。

在天地一体化网络中,缓存位置可以位于卫星、基站和网关。Xiangming Zhu 等人[115] 建立了一个三层协同缓存模型,并分析了不同缓存位置的命中概率和内容部署过程,并将内容部署问题建成为最小化用户平均内容获取延迟,设计了非协同缓存和协同缓存机制,利用迭代算法优化了缓存策略。此外,他[116] 针对不同区域用户的内容流行度分布和接入方式差异性,提出了一种三层缓存架构,引入时延边少增益和缓存分配权衡,分别考了不同区域站时延优先缓存策略,并进一步提出了动态规划算法和基于演化算法的局部缓存策略,实现为基站接入用户和卫星接入用户提供高质量的内容服务。Xiangqiang Gao 等人[117] 提出了基于邻域搜索的协同缓存和资源分配算法,通过优化存储资源使用和网络带宽消耗的部署成本,以降低通信延迟和带宽消耗。Haijun Zhang 等人[118] 提出了基于协同缓存技术的缓存放置和功率分配联合优化问题,考虑了缓存大小限制和功率约束,在提升系统性能和降低功率消耗方面具有良好的效果。Jiaran Zhang 等人[119] 提出了基于内容感知的缓存比例分配方案,通过捕捉内容的时变请求特征和预测卫星的移动性和数据传输方向,延长缓存服务在移动大规模 LEO 卫星星座上的驻留时间,提高了有限 LEO 卫星载荷资源下的内容传输效率。

信息中心网络(Information-Centric Networking,ICN)[120] 将内容作为网络交换和路由的核心,允许用户直接请求和获取内容,而不需要关心内容的存储位置。将 ICN 与卫星网络结合[121,122],能够实现内容在卫星节点间的智能分发与协同缓存,有效提升网络资源利用率与数据获取效率。Laura Galluccio 等人[123] 提出缓存方案 SatCache,利用卫星通信的广播特性,通过创建用户偏好配置文件来预测用户对特定内容的潜在兴趣。网络内缓存是 ICN 的功能特性,能减少地面链路卫星网络中的数据存储和分发。Rui Xu 等人[124] 提出了基于两层缓存模型的遥感卫星网络协作缓存策略,该模型在卫星网络和地面站分别设置缓存节点,实现了全球级内容的获取,以及根据业务特性将缓存分为专

Denby 等人[112] 提出了在轨边缘计算 OEC 架构,旨在减少对地面基础设施的依赖,从而缓解传统管控架构在星座规模扩大时面临的通信瓶颈问题。由于卫星的通信与计算资源具有时变性,合理选择服务实体的部署节点并在恰当时机完成迁移,对于保障用户体验质量具有重要影响。Qing Li 等人[113] 提出了在线服务放置算法,通过结合了 Lyapunov 优化和 Gibbs 采样方法提高资源受限情况下的服务覆盖率,解决了 LEO 卫星星座中 SEC 节点在服务覆盖与鲁棒性之间的冲突问题。Ziqi Li 等人[114] 根据卫星资源的时变性与可见时间限制,动态选择最优服务节点并适时迁移服务实体,并将该问题建模为马尔可夫决策过程,并提出了基于惩罚–近端策略优化的在线部署方案 COMPOSE,旨在在延迟约束与迁移成本之间实现平衡。

在天地一体化网络中,缓存位置可以位于卫星、基站和网关。Xiangming Zhu 等人[115] 建立了一个三层协同缓存模型,并分析了不同缓存位置的命中概率和内容部署过程,并将内容部署问题建成为最小化用户平均内容获取延迟,设计了非协同缓存和协同缓存机制,利用迭代算法优化了缓存策略。此外,他[116] 针对不同区域用户的内容流行度分布和接入方式差异性,提出了一种三层缓存架构,引入时延边少增益和缓存分配权衡,分别考了不同区域站时延优先缓存策略,并进一步提出了动态规划算法和基于演化算法的局部缓存策略,实现为基站接入用户和卫星接入用户提供高质量的内容服务。Xiangqiang Gao 等人[117] 提出了基于邻域搜索的协同缓存和资源分配算法,通过优化存储资源使用和网络带宽消耗的部署成本,以降低通信延迟和带宽消耗。Haijun Zhang 等人[118] 提出了基于协同缓存技术的缓存放置和功率分配联合优化问题,考虑了缓存大小限制和功率约束,在提升系统性能和降低功率消耗方面具有良好的效果。Jiaran Zhang 等人[119] 提出了基于内容感知的缓存比例分配方案,通过捕捉内容的时变请求特征和预测卫星的移动性和数据传输方向,延长缓存服务在移动大规模 LEO 卫星星座上的驻留时间,提高了有限 LEO 卫星载荷资源下的内容传输效率。

信息中心网络(Information-Centric Networking,ICN)[120] 将内容作为网络交换和路由的核心,允许用户直接请求和获取内容,而不需要关心内容的存储位置。将 ICN 与卫星网络结合[121,122],能够实现内容在卫星节点间的智能分发与协同缓存,有效提升网络资源利用率与数据获取效率。Laura Galluccio 等人[123] 提出缓存方案 SatCache,利用卫星通信的广播特性,通过创建用户偏好配置文件来预测用户对特定内容的潜在兴趣。网络内缓存是 ICN 的功能特性,能减少地面链路卫星网络中的数据存储和分发。Rui Xu 等人[124] 提出了基于两层缓存模型的遥感卫星网络协作缓存策略,该模型在卫星网络和地面站分别设置缓存节点,实现了全球级内容的获取,以及根据业务特性将缓存分为专可以看出,现有研究在卫星边缘缓存与协作策略方面已开展了深入探索,涵盖了服务实体的动态部署、缓存位置优化、内容流行度预测及基于强化学习的决策机制等方面,在提升传输效率、降低服务时延、缓解网络负载等方面取得了显著成效。然而,在面向大规模低轨卫星星座的实际应用中,仍面临诸多挑战,例如星间链路频繁变化带来的协作路径不稳定、有限的星上计算与存储能力对缓存策略设计的制约,以及任务驱动、多业务融合场景下对资源调度的时效性与灵活性提出的更高要求。因此,如何在多星覆盖场景中实现高效的资源管理是一个极大的挑战。

\section{大规模低轨星座分布式智能管控的核心挑战}

- 大规模高速动态拓扑挑战: LEO卫星(通常运行于300-2000km轨道高度)数量庞大(可达数千乃至上万颗)且以7.8km/s左右的高速运动,导致星间链路(ISL)与星地链路(GSL)的连通性频繁变化。卫星的高速运动引起显著的多普勒频移(Doppler Shift)与链路快速衰落(Fast Fading),同时,LEO卫星的窄波束覆盖(单波束覆盖半径通常数百公里)与高速运动特性导致地面用户设备(UE)需频繁执行卫星切换(Handover/HO),切换周期可低至数十秒级别。这种极高动态性对网络资源调度算法的实时性、自适应能力与鲁棒性提出了前所未有的挑战。
- 异构业务与严格QoS保障挑战: 6G需同时支持增强型移动宽带(eMBB)、超可靠低时延通信(uRLLC)和海量机器类通信(mMTC)等差异化特征显著的垂直行业应用。各类业务对吞吐量(Throughput)、时延(Latency)、可靠性(Reliability)、连接密度(Connection Density)等服务质量(QoS)指标要求各异且极为严苛。卫星链路固有的大传播时延、受限带宽和星上计算资源约束,叠加广域覆盖下用户分布不均匀、信道条件空间差异性大(spatial heterogeneity)的特点,使得在卫星平台上实现端到端差异化QoS保障(End-to-End Differentiated QoS Provisioning)难度陡增。
- 资源高度受限与传输可靠性挑战: LEO卫星受限于星载能源、散热能力与载荷尺寸,其通信资源(频谱、功率、计算能力)极其有限。单颗卫星广覆盖特性导致其单个波束内可能存在数百至上千个并发活跃用户,在接入侧、路由层面和拥塞控制层面均存在激烈的资源竞争。地面站(Gateway)的非均匀地理分布易造成部分星地回传链路(Feeder Link)成为流量瓶颈,引发拥塞。简单“贪婪式”资源调度策略常导致用户服务中断或性能严重劣化。此外,卫星链路易受阴影效应、大气衰减(尤其是Ka/Q/V频段)和空间气象影响,呈现高误码率(High BER)和链路不稳定特性。传统依赖重传(ARQ/HARQ)确保可靠性的机制在高丢包环境下效率低下,显著增加端到端时延和时延抖动(Jitter)。
大规模动态拓扑下的网络管控挑战:​大规模LEO星座(如数千颗卫星的组网)的管控面临集中式与分布式控制架构的两难困境。集中式控制依赖中心节点进行全局状态收集与优化决策,在高速动态、大传播时延的LEO环境中面临不可接受的响应延迟、巨大信令开销以及单点故障风险,可扩展性(Scalability)差。分布式控制虽具天然可扩展性,但LEO网络的高动态拓扑(Topology Dynamics)、频繁链路状态波动以及节点间资源(如轨道高度、载荷能力)的异构性(Heterogeneity),使得分布式智能体(Agents)仅基于局部观测(Local Observation)易陷入局部最优(局部“短视”)或策略冲突,缺乏有效的全局协同优化能力。

\section{论文研究内容与组织架构}




上海交通大学是我国历史最悠久的高等学府之一,是教育部直属、教育部与上海市共建的全
国重点大学,是国家“七五”、“八五”重点建设和“211 工程”、“985 工程”的首批建
设高校。经过 115 年的不懈努力,上海交通大学已经成为一所“综合性、研究型、国际化”
的国内一流、国际知名大学,并正在向世界一流大学稳步迈进。 

{\songti 十九世纪末,甲午战败,民族危难。中国近代著名实业家、教育家盛宣怀和一批
  有识之士秉持“自强首在储才,储才必先兴学”的信念,于 1896 年在上海创办了交通大
  学的前身——南洋公学。建校伊始,学校即坚持“求实学,务实业”的宗旨,以培养“第
  一等人才”为教育目标,精勤进取,笃行不倦,在二十世纪二三十年代已成为国内著名的
  高等学府,被誉为“东方MIT”。抗战时期,广大师生历尽艰难,移转租界,内迁重庆,
  坚持办学,不少学生投笔从戎,浴血沙场。解放前夕,广大师生积极投身民主革命,学校
  被誉为“民主堡垒”。
  
  新中国成立初期,为配合国家经济建设的需要,学校调整出相当一部分优势专业、师资设
  备,支持国内兄弟院校的发展。五十年代中期,学校又响应国家建设大西北的号召,根据
  国务院决定,部分迁往西安,分为交通大学上海部分和西安部分。1959 年 3 月两部分同
  时被列为全国重点大学,7 月经国务院批准分别独立建制,交通大学上海部分启用“上海
  交通大学”校名。历经西迁、两地办学、独立办学等变迁,为构建新中国的高等教育体
  系,促进社会主义建设做出了重要贡献。六七十年代,学校先后归属国防科工委和六机部
  领导,积极投身国防人才培养和国防科研,为“两弹一星”和国防现代化做出了巨大贡
  献。}

{\heiti 改革开放以来,学校以“敢为天下先”的精神,大胆推进改革:率先组成教授代
  表团访问美国,率先实行校内管理体制改革,率先接受海外友人巨资捐赠等,有力地推动
  了学校的教学科研改革。1984 年,邓小平同志亲切接见了学校领导和师生代表,对学校
  的各项改革给予了充分肯定。在国家和上海市的大力支持下,学校以“上水平、创一流”
  为目标,以学科建设为龙头,先后恢复和兴建了理科、管理学科、生命学科、法学和人文
  学科等。1999 年,上海农学院并入;2005 年,与上海第二医科大学强强合并。至此,学
  校完成了综合性大学的学科布局。近年来,通过国家“985 工程”和“211 工程”的建
  设,学校高层次人才日渐汇聚,科研实力快速提升,实现了向研究型大学的转变。与此同
  时,学校通过与美国密西根大学等世界一流大学的合作办学,实施国际化战略取得重要突
  破。1985 年开始闵行校区建设,历经 20 多年,已基本建设成设施完善,环境优美的现
  代化大学校园,并已完成了办学重心向闵行校区的转移。学校现有徐汇、闵行、法华、七
  宝和重庆南路(卢湾)5 个校区,总占地面积 4840 亩。通过一系列的改革和建设,学校
  的各项办学指标大幅度上升,实现了跨越式发展,整体实力显著增强,为建设世界一流大
  学奠定了坚实的基础。}

{\ifcsname fangsong\endcsname\fangsong\else[无 \cs{fangsong} 字体。]\fi 交通大学
  始终把人才培养作为办学的根本任务。一百多年来,学校为国家和社会培养了 20 余万各
  类优秀人才,包括一批杰出的政治家、科学家、社会活动家、实业家、工程技术专家和医
  学专家,如江泽民、陆定一、丁关根、汪道涵、钱学森、吴文俊、徐光宪、张光斗、黄炎
  培、邵力子、李叔同、蔡锷、邹韬奋、陈敏章、王振义、陈竺等。在中国科学院、中国工
  程院院士中,有 200 余位交大校友;在国家 23 位“两弹一星”功臣中,有 6 位交大校
  友;在 18 位国家最高科学技术奖获得者中,有 3 位来自交大。交大创造了中国近现代
  发展史上的诸多“第一”:中国最早的内燃机、最早的电机、最早的中文打字机等;新中国
  第一艘万吨轮、第一艘核潜艇、第一艘气垫船、第一艘水翼艇、自主设计的第一代战斗
  机、第一枚运载火箭、第一颗人造卫星、第一例心脏二尖瓣分离术、第一例成功移植同种
  原位肝手术、第一例成功抢救大面积烧伤病人手术等,都凝聚着交大师生和校友的心血智
  慧。改革开放以来,一批年轻的校友已在世界各地、各行各业崭露头角。}

{\ifcsname kaishu\endcsname\kaishu\else[无 \cs{kaishu} 字体。]\fi 截至 2011 年
  12 月 31 日,学校共有 24 个学院 / 直属系(另有继续教育学院、技术学院和国际教育
  学院),19 个直属单位,12 家附属医院,全日制本科生 16802 人、研究生 24495 人
  (其中博士研究生 5059 人);有专任教师 2979 名,其中教授 835 名;中国科学院院
  士 15 名,中国工程院院士 20 名,中组部“千人计划”49 名,“长江学者”95 名,国家杰
  出青年基金获得者 80 名,国家重点基础研究发展计划(973 计划)首席科学家 24 名,
  国家重大科学研究计划首席科学家 9 名,国家基金委创新研究群体 6 个,教育部创新团
  队 17 个。
  
  学校现有本科专业 68 个,涵盖经济学、法学、文学、理学、工学、农学、医学、管理学
  和艺术等九个学科门类;拥有国家级教学及人才培养基地 7 个,国家级校外实践教育基
  地 5 个,国家级实验教学示范中心 5 个,上海市实验教学示范中心 4 个;有国家级教
  学团队 8 个,上海市教学团队 15 个;有国家级教学名师 7 人,上海市教学名师 35
  人;有国家级精品课程 46 门,上海市精品课程 117 门;有国家级双语示范课程 7 门;
  2001、2005 和 2009 年,作为第一完成单位,共获得国家级教学成果 37 项、上海市教
  学成果 157 项。}
