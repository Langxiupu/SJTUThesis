% !TEX root = ../main.tex

\chapter{绪论}

\section{研究背景及意义}
\subsection{低轨卫星通信的快速发展与6G融合趋势}
近年来,全球通信网络正经历由地面系统向空天地一体化体系的深度演进。随着“新太空时代”的到来,卫星通信的角色已发生显著变化,从传统广播与地面移动通信的补充环节转向支撑全球数字化与智能化转型的关键基础设施\cite{sun-SatDevelopment-satapp2023}。低轨卫星(Low Earth Orbit, LEO)星座的兴起,使“广覆盖、低时延、高容量”的天地融合网络成为现实。其中,SpaceX发射的Starlink系统已成为代表性工程,截至2025年部署通信卫星数量超过7000颗,在轨通信能力约22 Tbps,预计到2030年将提升至208 Tbps,占全球在轨通信容量的约74\%\cite{wang-InternationalSatCom-satapp2025}。OneWeb、Kuiper及SES-Intelsat联合星座等项目的实施,进一步推动了低轨通信从技术验证向体系化部署的跃迁。我国亦将卫星互联网纳入“新型基础设施”布局,“国网”、“千帆”、“行云”、“鸿雁”等星座计划稳步推进,为未来空天地一体化通信体系奠定了工程基础\cite{lin-Satinternet-broadcast2023}。由此可见,低轨卫星网络已越过萌芽阶段,进入规模化部署与全球组网的新时期,成为未来通信体系中不可或缺的基础层。

第六代移动通信系统($6^{\text{th}}$ Generation Mobile Communication Technology, 6G)的提出,为低轨卫星通信带来了新的定位与目标。国际电信联盟(International Telecommunication Union, ITU)在2023年发布的6G愿景中,首次将“空天地一体化”(Space–Air–Ground Integrated Network, SAGIN)确立为核心架构要素,强调通过卫星网络实现“泛在智能连接”,以弥补地面系统在全球连续性与服务广度方面的局限\cite{wang-6GSATIntro-radioCom2024}。IMT-2030(6G)推进组进一步指出,低轨卫星网络将成为6G架构的原生组成部分,通过在协议、频谱与调度机制上的统一,实现非地面网络(Non-Terrestrial Network, NTN)与地面网络(Terrestrial Network, TN)的协同运行\cite{wang-Integrated6GSat-newcom2023}。这意味着卫星系统不再是单纯的补充接入层,而将直接参与6G体系的端到端服务质量(Quality of Service, QoS)保障与业务承载。6G的发展使卫星通信由“覆盖型网络”向“服务型网络”转变,业务需求的上升推动了卫星系统的体系演进。增强型移动宽带(enhanced Mobile Broadband, eMBB)对速率与容量的追求,使低轨星座持续向更高通量方向演化;超可靠低时延通信(Ultra-Reliable and Low-Latency Communications, uRLLC)要求在动态链路与频繁切换条件下保持稳定时延与可靠性;大规模机器通信(massive Machine-Type Communications, mMTC)则带来更高的接入密度与资源调度复杂度。这些新特征促使卫星通信的目标从“可连接”迈向“可保障”,成为6G实现全球无缝覆盖与多场景服务的一体化支撑。

为了应对6G对网络连续性、低时延与可扩展性的更高要求,低轨卫星网络正沿着星上处理增强、智能决策引入与管控架构演进的方向发展。星上处理与资源编排能力的持续提升,使卫星逐步具备在轨对通信、计算与缓存资源进行联合调度的能力;可重构载荷与数字透明转发技术的成熟,促进了星上业务识别与链路自适应优化\cite{fang-6GSatCompute-SpaceElec2023,zou-SatComSurvey-SpaceElec2023}。同时,人工智能(Artificial Intelligence, AI)的引入为卫星网络带来更强的环境感知与自适应能力,使卫星能够根据信道状态与业务分布进行动态调整,实现面向服务的精细化调度。随着星座规模持续扩大,网络管控模式也在由集中式控制向分布式协同演化,卫星节点通过轻量级信息交互与分层协调机制开展多星协同,实现跨节点的任务分解与资源联合优化\cite{wang-Integrated6GSat-newcom2023}。
总体而言,低轨卫星通信正从单一的链路传输系统发展为具备自治、协同与智能特征的网络体系,这一趋势反映了6G时代通信架构从“覆盖驱动”向“服务驱动”的整体转变,并为实现端到端QoS保障与全球一体化连接提供了持续的技术动力。

\subsection{迈向智能化与分布式协同的低轨星座演进趋势}
随着星座规模的持续扩大与业务类型的多样化,LEO 网络正由资源驱动的“可连接”体系迈向智能驱动的“可保障”体系。面对 6G 对灵活性、实时性与自治度的更高要求,其演进路径呈现出由星上智能化向分布式协同、再到智能融合自治的连续演化趋势\cite{zhang-distributedSat-SCIS2025,zhang-distributedSat-SCIS2025}。在此背景下,传统基于静态模型的集中优化方法难以兼顾实时性与全局性,促使 AI 成为应对 LEO 强动态特征的关键手段。随着星上处理与可重构载荷能力的提升,AI 已被用于信道预测、波束成形、功率控制与路由决策等任务。已有研究表明,多目标深度强化学习可在多波束场景下实现波束跳频与覆盖率的动态平衡\cite{hu-beamhopping-tob2020}。

LEO 网络以星座规模大、轨道高度低、链路时变快为主要特征,典型系统包含数千颗在轨卫星,节点高速运动导致拓扑频繁重构。同时,星地回程链路受带宽与可见时间限制,易出现链路不稳定与资源瓶颈,使集中式管控难以及时获取全局状态并执行最优决策


\section{国内外研究现状及分析}
\subsection{天地一体化网络架构}
5G 网络是地面无线通信的核心组成部分,在很大程度上满足了人们的服务需求。然而,由于网络覆盖范围及容量有限,在偏远地区仍难以提供高速率和高可靠性的通信服务。将卫星网络与地面网络的优势相结合,构建天地一体化网络架构是未来网络的重要发展趋势[35]。天地一体化网络融合了多种异构网络,为海陆天用户提供一个无缝的信息服务,能够满足未来网络在任何时间、任何地点的通信需求。

天地一体化网络是融合地面网络、空中平台与卫星网络的新型通信架构,旨在通过微波、太赫兹或激光链路实现不同网段间的信息互通,并通过星上处理、星上路由等技术手段,打通异构网络之间的互联瓶颈。其中,地面网络具备低时延、低功耗、高吞吐等优势,可为用户提供高速接入服务,并依托地面管理中心的强大计算与存储能力承担主要的数据处理与信息调度任务;蜂窝网络则为移动终端提供低时延通信支持。地面 5G 网络虽覆盖范围有限,但以约 1 ms 的超低时延与高吞吐性能,在部署成本方面具明显优势。空中平台包括 HAP 与低空平台(Low Altitude Platform,LAP),通常由 UAV、气球、飞机等高机动性设备组成,可用于增强特定区域的覆盖能力,尤其在应急通信场景中具有重要价值。空中平台部署高度介于 10–30 km 之间,RTT 低于 15 ms,具备较高的部署灵活性。但连接稳定性和数据承载能力存在一定局限。卫星网络主要由 GEO、MEO 与 LEO 卫星构成,具备广域覆盖的能力,但数据承载能力有限。GEO 卫星部署高度约为 35,786 km,虽然覆盖范围广,但往返时延(Round-Trip Time,RTT)超过 600 ms,部署成本极高;MEO 与 LEO 卫星在成本与时延方面更具优势,尤其是 LEO 卫星,其 RTT 可降低至 20–25 ms,适用于对时延敏感的业务需求。表 1-2 对天地一体化网络中不同网段的特征进行了系统总结。总体来看,各类网络在覆盖范围、时延特性、部署成本和通信能力等方面存在显著差异,对天地一体化网络的架构设计与资源调度策略提出了更高的要求。天地一体化网络架构设计是实现全球无缝覆盖与高效跨域互联的关键途径,已成为当前研究的重要方向。天地一体化网络架构属于异构网络,仍存在资源分配与通信协议优化等核心问题。Haipeng Yao 等人[36] 提出了天地一体化网络的潜在架构,并讨论了架构所面临的关键技术挑战。该架构整合了扩展的空间网络、互联网和移动无线网络,以提供全球全时全网访问的综合服务。Su Yao 等人[37] 提出了智慧标识网络 SI-STIN 架构,包括三层网域(智能普适服务层、动态资源适配层和协同网络组件层,以及实体域和行为域)的网络结构,解决了异构网络的融合问题,并打破了传统网络层设计的三重绑定,实现了资源整合和互联的高效性。Jiaxin Zhang 等人[38] 提出了利用多接入边缘计算和人工智能提升网络智能管理的 DILIGENT 架构,重点优化任务卸载与内容缓存分发。Renchao Xie 等人[39] 提出的 STENCN 架构则在多层架构边缘计算中优化了计算资源调度,解决了卫星仅作为中继网络的局限。Ting Ma 等人[40] 针对超密集 LEO 天地一体化网络的控制与管理问题,提出了融合 MEO 星、LEO 星和地面网星站的多层管理架构。该架构通过引入全局控制器与局部控制器,降低网络管理复杂度,支持高效的网络状态感知、移动性管理、资源调度与服务管理。Shuo Yuan 等人[41] 提出了面向 6G 的卫星–地面融合雾计算网络的系统架构及关键技术,重点解决 LEO 卫星在通信、遥感与导航等多任务融合场景下的计算与协同能力不足问题,并探讨了如集成波形设计、资源管理、移动性管理及云–雾协同下的原生人工智能等核心技术方案。

作为一种新型网络设计理念,软件定义网络(Software-Defined Networking, SDN)在网络状态监测和跨域资源调度等方面具备显著优势。基于 SDN 的天地一体化网络架构通过智能管理与编排系统实现对网络基础设施的高效管理,其中 SDN 控制器的部署方式是研究的关键重点。控制器部署方案包括集中式与分层式两种方式,如表 1-3 所示。

集中式控制器通常部署在地面或卫星上,进行整个网络的集中管理。Bowei Yang 等人[42] 提出的 SDSN 架构将 SDN 控制器部署在地面,利用 GEO 卫星转发控制指令来配置 LEO 卫星,从而实现网络配置的高效管理。该方法简化了不同卫星之间的网络互联,并便于新协议的测试与部署。Yuanguo Bi 等人[43] 也提出将 SDN 控制器部署在地面网络上,用于收集全网信息。Jinzhen Bao 等人[44] 提出的 OpenSAN 的架构将 SDN 控制器部署在 GEO 卫星上,所有路由转发策略皆通过 GEO 卫星进行。Zhenning Zhang 等人[45] 提出的单层 LEO 卫星网络控制框架,避免了由 GEO 卫星转发[44] 造成的瓶颈问题。随着卫星网络规模扩大,LEO 卫星的高移动性和 ISL 的传播延迟,集中式控制器在处理来自数据平面的请求时遇到瓶颈。

分层部署方案则是将控制器部署在多层网络中,使同层控制器之间能够协同交换信息,从而实现更高效的网络管理与资源调度。Taixin Li 等人[46] 基于 SDN 和网络功能虚拟化(Network Functions Virtualization, NFV)架构提出了 SERvICE 架构,该架构旨在实现灵活的卫星网络配置和细粒度的 QoS 保证。SERvICE 架构的控制器部署在地面数据中心以及卫星网关和 GEO 卫星上,负责传输 GEO 卫星收集的网络信息到地面管理中心,以及流表下发与流量控制。Yongpeng Shi 等人[47] 提出的 MLSTIN 架构分别在 GEO 卫星、HAP 和地面网络中部署控制器。Bohao Feng 等人[48] 提出的 HetNet 网络架构结合了定位/ID 分离和信息中心网络技术,实现异构网络融合,路由可扩展性缓解、移动性支持、流量工程和高效内容传输。此外,HetNet 也采用 SDN 和 NFV 技术,以进一步提升网络弹性。Min Sheng 等人[49] 提出了灵活资源共享卫星网络 FRBSN 架构,通过 SDN 和 NFV 技术克服传统卫星网络资源利用不足的问题,并利用资源演化图描述多维资源在卫星网络中的演化关系。

随着 5G 的发展,针对特定应用场景的天地一体化网络架构也被相继提出。在车联网(Internet of Vehicle, IoV)场景中,Ning Zhang 等人[50] 提出分层结构的软件定义网络架构 SSAGV,通过网络切片保护卫星、空中和地面段的传输服务,并将资源集中到一个动态资源池中,由分层控制器进行管理,以支持车载服务。Xiaoya Zhang 等人[51] 提出了支持的车联网的天地一体化架构 SGIIN-IoV,该架构支持多链路接入、高延迟容忍数据通过卫星链路传输,并通过车内服务层、链路检测层和全球监控层实现 SGIIN-IoV 的全维度监控。在物联网(Internet of Things, IoT)场景中,Wei-Che Chien 等人[52] 提出基于物联网、移动网络和卫星网络发展趋势的 H-STIN 架构,结合物联网、SDN 和 NFV 技术,解决大带宽需求、大规模机器类型通信和无边缘通信的挑战。Tao Hong 等人[53] 介绍了 SAG IoT 网络范式及其组成和网络架构,强调了网络切片技术在 SAG IoT 中的应用、无人机对毫米波频道的影响,以及机器学习在 SAG IoT 中的用途。Tianhao Liang 等人[54] 提出了基于无人机–卫星协同的架构,用于实现物联网场景下的一体化定位与通信服务,其中无人机作为空中的中继节点,卫星则承担定位基础与通信传功能,探讨了定位算法、波束赋形、帧结构设计、无人机部署与轨迹规划等关键技术,并指出了同步、频谱共享和硬件受限等研究挑战。Yun Hu 等人[55] 聚焦于面向实时物联网应用的卫星网络时效性保障问题,构建了基于卫星的物联网整体架构,并围绕时间变化的网络建模与确定性控制方法展开关键技术探讨,提出了适用于卫星上计算任务和数据密集型服务的时效性路由方案,展望了卫星物联网未来的研究方向。在自然灾害或大规模突发事件等应急场景中,传统地面网络可能会过载、瘫痪或完全摧毁。为应对上述场景,Ying Wang 等人[56] 提出的混合天地一体化网络 HSAT 架构,讨论了 HAP 在应急网络中的应用及相关技术。

\subsection{用户切换策略}
SDN 将控制平面和数据平面分离,实现了网络配置的可编程性,提供了更灵活的网络管理方式[57]。在 SDN 网络中,控制器是负责管理和配置网络设备的核心。控制器部署问题(Controller Placement Problem,CPP)包括 SDN 控制器的物理位置、数量和角色,其设计直接影响网络的性能和效率。CPP 可以视为一个单目标或多目标的优化问题,通常包括最小化网络延迟、降低维护成本、提高网络可靠性等关键目标[58,59]。

在传统网络架构中,控制器的分配通常是固定的,缺乏动态调整能力。在数据中心网络中,由于交换机被静态地分配给控制器,控制器的静态配置可能会因流量动态变化导致响应时间延长和维护成本增加。随着网络流量的波动,故障的发生以及控制器负载的不均衡,静态分配难以满足实时性与效率要求。因此,支持控制器动态分配机制对于提升网络适应性与管理效率具有重要意义。在动态控制器分配中,Tao Wang 等人[60] 通过在线优化方法来最小化响应时间和维护成本,将控制器分配问题划为一个稳定匹配问题,使用户随机地匹配到域控制框架优化控制器的分配。Samaresh Bera 等人[61] 提出了基于流量特定要求的动态控制器分配方案,通过自适应阈值选择和动态稳定匹配博弈来进行流量和控制器之间的分配,以减少流量时延和控制开销,并提高流量的 QoS 保证。Ashutosh Kumar Singh 等人[62] 提出了基于 Varma 优化的控制器部署问题求解方法,该方法旨在最小化网络的平均延迟。此外,还研究了控制器部署方案的能耗,包括部署成本和能耗,以实现绿色经济。Ilora Maity 等人[63] 提出了基于 IoT 流量的能量感知控制器部署方案,在带内控制平面下设计合理的控制器部署和路径选择来减少能量消耗。Alejandro Ruiz-Rivera 等人[64] 提出了降低控制器部署能耗算法,通过关闭尽可能多的链路来节省能量,同时考虑延迟、链路和控制器负载等约束。但是控制路径的重新路由可能会导致控制器过载。Ying Hong 等人[65] 提出了降低能量消耗的控制器部署方法,提出在时延和控制器负载的约束下最小化网络的能量消耗的二进制整数规划问题,并设计了遗传启发式算法对该问题进行求解。Adriana Fernandez-Fernandez 等人[66] 提出面向能效优化的流量工程方法,通过减少满足特定流量需求所需的链路数量,从而降低 SDN 网络中的能量消耗。其他研究还提出了多个控制器负载分布式管理,针对网络延迟、可靠性和负载平衡等多目标问题,提出了控制器部署的多维优化方法[67]。在控制器部署的优化算法方面,常采用线性规划[68]、贪心算法[69] 和模拟退火算法(Simulated Annealing Algorithm,SAA)[70] 等启发式方法,以求解不同场景下的最优控制器部署方案。

然而,以上研究多数集中在静态场景下的动态控制器部署方案。在卫星网络中,控制器部署面临更复杂的挑战。由于卫星时刻处于运动动态中,频繁地动态场景下的控制器部署策略[71]。在软件定义的卫星或天地一体化网络架构中,控制器的部署位置和数量,会影响整个网络系统的控制方式和管理策略。控制器通常部署在卫星网络和地面网络中,其选择直接影响到网络的管理效率、控制延迟和负载均衡能力。不同的控制器部署优化目标决定了总体网络的性能。此外,网络控制在大规模卫星星座中起着至关重要的作用,通过控制器协调大量网络节点,以保障未来空间通信网络的运行和服务的有效性和可靠性[72]。表 1-4 对比了软件定义卫星网络中不同控制器优化策略。

地面网络具有强大的计算能力和存储能力,地面网络与卫星网络的信息交
换即由卫星网关负责。Jiajia Liu 等人[73] 提出了针对软件定义的天地一体化网络中控制器和卫星网关部署问题的研究,探讨了卫星网络的部署问题以最小化平均延迟,并提出了基于模拟退火的近似解决方案。在控制器和网关的联合部署问题中,谋取令北区网络可靠性并满足延迟约束目标,采用基于模拟退火与聚类结合的混合算法对问题进行求解。Hua Qu 等人[74] 主要研究了在 UAV 上的控制器动态部署问题,以实现控制器的全球范围部署。Yongpeng Shi 等人[80] 讨论了天地一体化网络中的跨层网关选择问题,并将其定义为一个约束优化问题,利用贪心算法从空中网络中选取最优网关集合,作为连接地面层与卫星层的数据传输中继。Yizhou Shen 等人[81] 联合部署网关和控制器的问题,在改进的密度峰值聚类基础上,引入模拟退火算法以最小化负载平均差异率,并确定合理的控制器与网关数量。总
体上,地面控制器属于静态控制器,能耗与时延较长,难以满足动态流量需求。

由于卫星网络使用相对较低的延迟,控制器可以部署在 GEO 或 LEO 卫星上。Ariel Papa 等人[75] 考虑了用户的地理位置和时区变化对流量需求的动态影响,利用整数线性规划(Integer Linear Programming,ILP)来最小化流量延迟。LEO 卫星网络的流量分布与特定地理区域的活动密切相关[83]。例如,人口密度较高的区域往往会面临更为集中的通信需求。这种需求集中不仅导致卫星网络中相关区域的流量激增[84],而且也进一步加剧了网络资源的紧张。由于流量需求的不均匀分布,导致链路存在流量过载等情况。在卫星网络中,流量工程(Traffic Engineering,TE)是高效优化网络流量的重要手段[85]。

\subsection{动态切片策略}
LEO 卫星网络的流量分布与特定地理区域的活动密切相关[83]。例如,人口密度较高的区域往往会面临更为集中的通信需求。这种需求集中不仅导致卫星网络中相关区域的流量激增[84],而且也进一步加剧了网络资源的紧张。由于流量需求的不均匀分布,导致链路存在流量过载等情况。在卫星网络中,流量工程(Traffic Engineering,TE)是高效优化网络流量的重要手段[85]。

卫星网络的负载均衡算法可分为全局负载均衡路由算法与局部负载均衡路由算法。Xia Deng 等人[86] 提出了限制传输时长的路由方法,通过结合卫星之间的欧几里得距离和背压路由,将每个流的传输限制在特定区域内以减少传输冗余,并利用低拥塞的路径进行动态传输,以有效平衡整个网络的流量负载。Jiang Liu 等人[87] 提出了选择性分裂负载均衡路由算法,旨在通过降低低优先级链路的使用频率实现负载均衡。为此,他们改进了选择性迭代 Dijkstra 算法,以减少节点重复率并优化路径计算过程;同时,采用选择性拆分策略,将流量从拥塞节点引导至邻近节点,有效降低了计算资源需求与信令交互量。Guanghua Song 等人[88] 提出了一种基于交通灯的智能路由策略 TLR,该方法利用一组交通灯指标记录当前节点与下一跳节点的拥塞状态。数据包在被预定路径传输的过程中,可根据中继节点的实时交通灯颜色动态调整路由路径,实现了“预先规划与实时调整”的有效结合,从而为每个数据包选择近似最优的传输路径。TLR 采用局部负载均衡的路由算法,控制信息开销较小,但由于不能有效规避拥塞区域,易导致数据包被反复转发至同一拥塞区域,从而引发数据拥塞问题。全局负载均衡可以动态调整网络资源的分配,避免某些节点或链路过载,提升整个网络的资源利用率。Tarik Taleb 等人[89] 提出了显式式负载均衡算法 ELB,用于解决非静止卫星通信系统中因用户分布不均导致的链路拥塞问题。ELB 算法在卫星之间同步拥塞状态信息,在即将发生拥塞时,卫星可以请求邻近卫星来减缓数据转发速率,并通过寻找其他较少拥塞的路径传输数据,避免了卫星拥塞和数据丢包,实现了更好的流量分配。ELB 采用分布式优化方法,但它又依赖于不同节点之间的通信,可以视为全局负载均衡的方法。Yong Lu 等人[90] 提出了分布流量平衡路由的方法,保障了路由的生存性并提供流量平衡能力,还有效减少了由于卫星故障信息泛洪带来的计算和存储开销。Chunxiao Liu 等人[91] 利用剩余负载率进行混合路由,并将其与遗传算法和蚁群算法相结合,有效地平衡了网络流量,解决了 LEO 卫星网络中的负载均衡问题。

现有卫星网络的 TE 解决方案面临着可扩展性和复杂性相关的问题。分段路由(Segment Routing, SR)是一种新型的网络路由架构,利用多段路径的方式,使得网络中的每个路由决策都可以预先定义,并通过标志堆叠传递,能够有效提升网络流量管理的灵活性和效率[92,93]。随着分段路径的增加入,合并应增
加开销的成本,但有更灵活的路径选择。段段的搜索空间较小,快速调整和重整[95] 并基于机器学习的流量预测方法[96] 等可提高复杂动态环境中的流量波动和拥塞问题,显著提升流量工程的效率和灵活性。在 LEO 卫星网络中,Wei Liu 等人[97] 提出了基于分段控制的流量调度自适应路由算法,根据网关之间的位置和反向时隙动态划分载区和重载区,在重载区的感知层节点的载载延时、负载区和重载区采用了不同的路由规则,解决了由于地面网关分布在有限区域内而引起的网络拥塞问题。为了解决卫星 Internet 网络,Menghan Wu[98,99] 等人提出在 LEO 卫星网络中优化 ISL 性能的方法。对于广播传输场景,采用了基于内容分发的路由策略;而对于多播传输场景,利用了矩形扇射树构和链路负荷的斯坦纳树约提高带宽利用率。此外,为了应对卫星网络中的意外故障,SR 的快速重路由机制可以快速响应网络问题。Shengyu Zhang 等人[100] 提出了基于分段路由的流量拆分算法,并研究了两种快速重路由机制,以保证数据包在网络中的平衡转发和快速恢复。Xun Chen 等人[101] 提出了结合集中计算与分段路由的混合快速重路由方案,卫星可以预计算重路由路径,并将路由规则分发给具备缓存发送资源的卫星。

在卫星网络中,源节点和目的节点之间通常存在多条路径。为提高资源利用率与网络灵活性,引入多路径路由技术能够有效提供更高效的传输方案[102,103]。传统的等价多路径(Equal-Cost Multi-Path,ECMP)路由[104] 根据数据包报头的某些元素进行静态流量分割以实现路径调度,但没有考虑节点和延迟对网络参数的限制,仍会发生网络拥塞。在 SDN 网络中,Yurii Kulakov 等人[105] 提出由 SDN 控制器集中生成路由信息,并结合多路径路由策略,进一步提升动态流量重配置的效率。Md. Sajid Hossen[106] 使用了链路带宽消耗作为路径权重,并通过深度优先搜索(Depth First Search,DFS)选择最佳路径,在实现多路径负载均衡的同时提高了 SDN 的 QoS 保证能力。多层卫星网络由于其广泛的覆盖范围和高网络容量,可以构建高效的通信网络。由于流量拥塞,多层卫星网络同样可能会出现吞吐下降和严重的端到端延迟等情况。Wenchao Xu 等人[107] 提出了基于网络编码的多径路由算法,通过混合卫星网络传输流量,以有效满足多种数据流量的传输需求。Xiaoylu Liang 等人[109] 提出了基于深度强化学习(Deep Reinforcement Learning,DRL)的路由优化算法框架用于大规模卫星动态多路径复用策略,提高 LEO 卫星网络中的路径发现准确性和多路径转发性能。

可以看出,现有研究围绕卫星网络的流量工程与路由优化问题,已提出多种具有针对性的算法与策略,覆盖局端与全局负载均衡、分段路由、多路径传输等多个方面,并逐步引入了基于 SDN 的创新方法,显著提升了网络的灵活性与资源利用效率。然而,面对卫星网络的高时空动态特征和拓扑不均衡的快速重组带来的复杂调度约束问题,现有方法在全局最优性、实时调度能力和跨层协同机制方面仍存在不足,仍需进一步关注大规模 LEO 卫星网络的多路径优化与路由决策机制,以实现更高效、更鲁棒的卫星网络流量管理体系。

\subsection{分布式路由技术}
卫星边缘缓存是提升内容传输服务质量的关键手段。在卫星网络中部署边缘计算与存储功能,可使内容更接近用户侧,既显著降低数据传输时延,又有效减轻地面网络负载[110]。每颗卫星除了充当网络路由器外,也具备类似云计算功能,使 LEO 卫星星座具备在轨处理功能,能够根据服务的提供算力。Debopam Bhattacherjee 等人[111] 分析了卫星边缘计算(Satellite Edge Computing,SEC)模式可以在内容分发、边缘计算、多人游戏等领域带来潜在的益处。Bradkey Denby 等人[112] 提出了在轨边缘计算 OEC 架构,旨在减少对地面基础设施的依赖,从而缓解传统管控架构在星座规模扩大时面临的通信瓶颈问题。由于卫星的通信与计算资源具有时变性,合理选择服务实体的部署节点并在恰当时机完成迁移,对于保障用户体验质量具有重要影响。Qing Li 等人[113] 提出了在线服务放置算法,通过结合了 Lyapunov 优化和 Gibbs 采样方法提高资源受限情况下的服务覆盖率,解决了 LEO 卫星星座中 SEC 节点在服务覆盖与鲁棒性之间的冲突问题。Ziqi Li 等人[114] 根据卫星资源的时变性与可见时间限制,动态选择最优服务节点并适时迁移服务实体,并将该问题建模为马尔可夫决策过程,并提出了基于惩罚–近端策略优化的在线部署方案 COMPOSE,旨在在延迟约束与迁移成本之间实现平衡。

在天地一体化网络中,缓存位置可以位于卫星、基站和网关。Xiangming Zhu 等人[115] 建立了一个三层协同缓存模型,并分析了不同缓存位置的命中概率和内容部署过程,并将内容部署问题建成为最小化用户平均内容获取延迟,设计了非协同缓存和协同缓存机制,利用迭代算法优化了缓存策略。此外,他[116] 针对不同区域用户的内容流行度分布和接入方式差异性,提出了一种三层缓存架构,引入时延边少增益和缓存分配权衡,分别考了不同区域站时延优先缓存策略,并进一步提出了动态规划算法和基于演化算法的局部缓存策略,实现为基站接入用户和卫星接入用户提供高质量的内容服务。Xiangqiang Gao 等人[117] 提出了基于邻域搜索的协同缓存和资源分配算法,通过优化存储资源使用和网络带宽消耗的部署成本,以降低通信延迟和带宽消耗。Haijun Zhang 等人[118] 提出了基于协同缓存技术的缓存放置和功率分配联合优化问题,考虑了缓存大小限制和功率约束,在提升系统性能和降低功率消耗方面具有良好的效果。Jiaran Zhang 等人[119] 提出了基于内容感知的缓存比例分配方案,通过捕捉内容的时变请求特征和预测卫星的移动性和数据传输方向,延长缓存服务在移动大规模 LEO 卫星星座上的驻留时间,提高了有限 LEO 卫星载荷资源下的内容传输效率。

信息中心网络(Information-Centric Networking,ICN)[120] 将内容作为网络交换和路由的核心,允许用户直接请求和获取内容,而不需要关心内容的存储位置。将 ICN 与卫星网络结合[121,122],能够实现内容在卫星节点间的智能分发与协同缓存,有效提升网络资源利用率与数据获取效率。Laura Galluccio 等人[123] 提出缓存方案 SatCache,利用卫星通信的广播特性,通过创建用户偏好配置文件来预测用户对特定内容的潜在兴趣。网络内缓存是 ICN 的功能特性,能减少地面链路卫星网络中的数据存储和分发。Rui Xu 等人[124] 提出了基于两层缓存模型的遥感卫星网络协作缓存策略,该模型在卫星网络和地面站分别设置缓存节点,实现了全球级内容的获取,以及根据业务特性将缓存分为专

Denby 等人[112] 提出了在轨边缘计算 OEC 架构,旨在减少对地面基础设施的依赖,从而缓解传统管控架构在星座规模扩大时面临的通信瓶颈问题。由于卫星的通信与计算资源具有时变性,合理选择服务实体的部署节点并在恰当时机完成迁移,对于保障用户体验质量具有重要影响。Qing Li 等人[113] 提出了在线服务放置算法,通过结合了 Lyapunov 优化和 Gibbs 采样方法提高资源受限情况下的服务覆盖率,解决了 LEO 卫星星座中 SEC 节点在服务覆盖与鲁棒性之间的冲突问题。Ziqi Li 等人[114] 根据卫星资源的时变性与可见时间限制,动态选择最优服务节点并适时迁移服务实体,并将该问题建模为马尔可夫决策过程,并提出了基于惩罚–近端策略优化的在线部署方案 COMPOSE,旨在在延迟约束与迁移成本之间实现平衡。

在天地一体化网络中,缓存位置可以位于卫星、基站和网关。Xiangming Zhu 等人[115] 建立了一个三层协同缓存模型,并分析了不同缓存位置的命中概率和内容部署过程,并将内容部署问题建成为最小化用户平均内容获取延迟,设计了非协同缓存和协同缓存机制,利用迭代算法优化了缓存策略。此外,他[116] 针对不同区域用户的内容流行度分布和接入方式差异性,提出了一种三层缓存架构,引入时延边少增益和缓存分配权衡,分别考了不同区域站时延优先缓存策略,并进一步提出了动态规划算法和基于演化算法的局部缓存策略,实现为基站接入用户和卫星接入用户提供高质量的内容服务。Xiangqiang Gao 等人[117] 提出了基于邻域搜索的协同缓存和资源分配算法,通过优化存储资源使用和网络带宽消耗的部署成本,以降低通信延迟和带宽消耗。Haijun Zhang 等人[118] 提出了基于协同缓存技术的缓存放置和功率分配联合优化问题,考虑了缓存大小限制和功率约束,在提升系统性能和降低功率消耗方面具有良好的效果。Jiaran Zhang 等人[119] 提出了基于内容感知的缓存比例分配方案,通过捕捉内容的时变请求特征和预测卫星的移动性和数据传输方向,延长缓存服务在移动大规模 LEO 卫星星座上的驻留时间,提高了有限 LEO 卫星载荷资源下的内容传输效率。

信息中心网络(Information-Centric Networking,ICN)[120] 将内容作为网络交换和路由的核心,允许用户直接请求和获取内容,而不需要关心内容的存储位置。将 ICN 与卫星网络结合[121,122],能够实现内容在卫星节点间的智能分发与协同缓存,有效提升网络资源利用率与数据获取效率。Laura Galluccio 等人[123] 提出缓存方案 SatCache,利用卫星通信的广播特性,通过创建用户偏好配置文件来预测用户对特定内容的潜在兴趣。网络内缓存是 ICN 的功能特性,能减少地面链路卫星网络中的数据存储和分发。Rui Xu 等人[124] 提出了基于两层缓存模型的遥感卫星网络协作缓存策略,该模型在卫星网络和地面站分别设置缓存节点,实现了全球级内容的获取,以及根据业务特性将缓存分为专可以看出,现有研究在卫星边缘缓存与协作策略方面已开展了深入探索,涵盖了服务实体的动态部署、缓存位置优化、内容流行度预测及基于强化学习的决策机制等方面,在提升传输效率、降低服务时延、缓解网络负载等方面取得了显著成效。然而,在面向大规模低轨卫星星座的实际应用中,仍面临诸多挑战,例如星间链路频繁变化带来的协作路径不稳定、有限的星上计算与存储能力对缓存策略设计的制约,以及任务驱动、多业务融合场景下对资源调度的时效性与灵活性提出的更高要求。因此,如何在多星覆盖场景中实现高效的资源管理是一个极大的挑战。

\section{大规模低轨星座分布式智能管控的核心挑战}

- 大规模高速动态拓扑挑战: LEO卫星(通常运行于300-2000km轨道高度)数量庞大(可达数千乃至上万颗)且以7.8km/s左右的高速运动,导致星间链路(ISL)与星地链路(GSL)的连通性频繁变化。卫星的高速运动引起显著的多普勒频移(Doppler Shift)与链路快速衰落(Fast Fading),同时,LEO卫星的窄波束覆盖(单波束覆盖半径通常数百公里)与高速运动特性导致地面用户设备(UE)需频繁执行卫星切换(Handover/HO),切换周期可低至数十秒级别。这种极高动态性对网络资源调度算法的实时性、自适应能力与鲁棒性提出了前所未有的挑战。
- 异构业务与严格QoS保障挑战: 6G需同时支持增强型移动宽带(eMBB)、超可靠低时延通信(uRLLC)和海量机器类通信(mMTC)等差异化特征显著的垂直行业应用。各类业务对吞吐量(Throughput)、时延(Latency)、可靠性(Reliability)、连接密度(Connection Density)等服务质量(QoS)指标要求各异且极为严苛。卫星链路固有的大传播时延、受限带宽和星上计算资源约束,叠加广域覆盖下用户分布不均匀、信道条件空间差异性大(spatial heterogeneity)的特点,使得在卫星平台上实现端到端差异化QoS保障(End-to-End Differentiated QoS Provisioning)难度陡增。
- 资源高度受限与传输可靠性挑战: LEO卫星受限于星载能源、散热能力与载荷尺寸,其通信资源(频谱、功率、计算能力)极其有限。单颗卫星广覆盖特性导致其单个波束内可能存在数百至上千个并发活跃用户,在接入侧、路由层面和拥塞控制层面均存在激烈的资源竞争。地面站(Gateway)的非均匀地理分布易造成部分星地回传链路(Feeder Link)成为流量瓶颈,引发拥塞。简单“贪婪式”资源调度策略常导致用户服务中断或性能严重劣化。此外,卫星链路易受阴影效应、大气衰减(尤其是Ka/Q/V频段)和空间气象影响,呈现高误码率(High BER)和链路不稳定特性。传统依赖重传(ARQ/HARQ)确保可靠性的机制在高丢包环境下效率低下,显著增加端到端时延和时延抖动(Jitter)。
大规模动态拓扑下的网络管控挑战:​大规模LEO星座(如数千颗卫星的组网)的管控面临集中式与分布式控制架构的两难困境。集中式控制依赖中心节点进行全局状态收集与优化决策,在高速动态、大传播时延的LEO环境中面临不可接受的响应延迟、巨大信令开销以及单点故障风险,可扩展性(Scalability)差。分布式控制虽具天然可扩展性,但LEO网络的高动态拓扑(Topology Dynamics)、频繁链路状态波动以及节点间资源(如轨道高度、载荷能力)的异构性(Heterogeneity),使得分布式智能体(Agents)仅基于局部观测(Local Observation)易陷入局部最优(局部“短视”)或策略冲突,缺乏有效的全局协同优化能力。

\section{论文研究内容与组织架构}




上海交通大学是我国历史最悠久的高等学府之一,是教育部直属、教育部与上海市共建的全
国重点大学,是国家“七五”、“八五”重点建设和“211 工程”、“985 工程”的首批建
设高校。经过 115 年的不懈努力,上海交通大学已经成为一所“综合性、研究型、国际化”
的国内一流、国际知名大学,并正在向世界一流大学稳步迈进。 

{\songti 十九世纪末,甲午战败,民族危难。中国近代著名实业家、教育家盛宣怀和一批
  有识之士秉持“自强首在储才,储才必先兴学”的信念,于 1896 年在上海创办了交通大
  学的前身——南洋公学。建校伊始,学校即坚持“求实学,务实业”的宗旨,以培养“第
  一等人才”为教育目标,精勤进取,笃行不倦,在二十世纪二三十年代已成为国内著名的
  高等学府,被誉为“东方MIT”。抗战时期,广大师生历尽艰难,移转租界,内迁重庆,
  坚持办学,不少学生投笔从戎,浴血沙场。解放前夕,广大师生积极投身民主革命,学校
  被誉为“民主堡垒”。
  
  新中国成立初期,为配合国家经济建设的需要,学校调整出相当一部分优势专业、师资设
  备,支持国内兄弟院校的发展。五十年代中期,学校又响应国家建设大西北的号召,根据
  国务院决定,部分迁往西安,分为交通大学上海部分和西安部分。1959 年 3 月两部分同
  时被列为全国重点大学,7 月经国务院批准分别独立建制,交通大学上海部分启用“上海
  交通大学”校名。历经西迁、两地办学、独立办学等变迁,为构建新中国的高等教育体
  系,促进社会主义建设做出了重要贡献。六七十年代,学校先后归属国防科工委和六机部
  领导,积极投身国防人才培养和国防科研,为“两弹一星”和国防现代化做出了巨大贡
  献。}

{\heiti 改革开放以来,学校以“敢为天下先”的精神,大胆推进改革:率先组成教授代
  表团访问美国,率先实行校内管理体制改革,率先接受海外友人巨资捐赠等,有力地推动
  了学校的教学科研改革。1984 年,邓小平同志亲切接见了学校领导和师生代表,对学校
  的各项改革给予了充分肯定。在国家和上海市的大力支持下,学校以“上水平、创一流”
  为目标,以学科建设为龙头,先后恢复和兴建了理科、管理学科、生命学科、法学和人文
  学科等。1999 年,上海农学院并入;2005 年,与上海第二医科大学强强合并。至此,学
  校完成了综合性大学的学科布局。近年来,通过国家“985 工程”和“211 工程”的建
  设,学校高层次人才日渐汇聚,科研实力快速提升,实现了向研究型大学的转变。与此同
  时,学校通过与美国密西根大学等世界一流大学的合作办学,实施国际化战略取得重要突
  破。1985 年开始闵行校区建设,历经 20 多年,已基本建设成设施完善,环境优美的现
  代化大学校园,并已完成了办学重心向闵行校区的转移。学校现有徐汇、闵行、法华、七
  宝和重庆南路(卢湾)5 个校区,总占地面积 4840 亩。通过一系列的改革和建设,学校
  的各项办学指标大幅度上升,实现了跨越式发展,整体实力显著增强,为建设世界一流大
  学奠定了坚实的基础。}

{\ifcsname fangsong\endcsname\fangsong\else[无 \cs{fangsong} 字体。]\fi 交通大学
  始终把人才培养作为办学的根本任务。一百多年来,学校为国家和社会培养了 20 余万各
  类优秀人才,包括一批杰出的政治家、科学家、社会活动家、实业家、工程技术专家和医
  学专家,如江泽民、陆定一、丁关根、汪道涵、钱学森、吴文俊、徐光宪、张光斗、黄炎
  培、邵力子、李叔同、蔡锷、邹韬奋、陈敏章、王振义、陈竺等。在中国科学院、中国工
  程院院士中,有 200 余位交大校友;在国家 23 位“两弹一星”功臣中,有 6 位交大校
  友;在 18 位国家最高科学技术奖获得者中,有 3 位来自交大。交大创造了中国近现代
  发展史上的诸多“第一”:中国最早的内燃机、最早的电机、最早的中文打字机等;新中国
  第一艘万吨轮、第一艘核潜艇、第一艘气垫船、第一艘水翼艇、自主设计的第一代战斗
  机、第一枚运载火箭、第一颗人造卫星、第一例心脏二尖瓣分离术、第一例成功移植同种
  原位肝手术、第一例成功抢救大面积烧伤病人手术等,都凝聚着交大师生和校友的心血智
  慧。改革开放以来,一批年轻的校友已在世界各地、各行各业崭露头角。}

{\ifcsname kaishu\endcsname\kaishu\else[无 \cs{kaishu} 字体。]\fi 截至 2011 年
  12 月 31 日,学校共有 24 个学院 / 直属系(另有继续教育学院、技术学院和国际教育
  学院),19 个直属单位,12 家附属医院,全日制本科生 16802 人、研究生 24495 人
  (其中博士研究生 5059 人);有专任教师 2979 名,其中教授 835 名;中国科学院院
  士 15 名,中国工程院院士 20 名,中组部“千人计划”49 名,“长江学者”95 名,国家杰
  出青年基金获得者 80 名,国家重点基础研究发展计划(973 计划)首席科学家 24 名,
  国家重大科学研究计划首席科学家 9 名,国家基金委创新研究群体 6 个,教育部创新团
  队 17 个。
  
  学校现有本科专业 68 个,涵盖经济学、法学、文学、理学、工学、农学、医学、管理学
  和艺术等九个学科门类;拥有国家级教学及人才培养基地 7 个,国家级校外实践教育基
  地 5 个,国家级实验教学示范中心 5 个,上海市实验教学示范中心 4 个;有国家级教
  学团队 8 个,上海市教学团队 15 个;有国家级教学名师 7 人,上海市教学名师 35
  人;有国家级精品课程 46 门,上海市精品课程 117 门;有国家级双语示范课程 7 门;
  2001、2005 和 2009 年,作为第一完成单位,共获得国家级教学成果 37 项、上海市教
  学成果 157 项。}
