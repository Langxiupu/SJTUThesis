% !TEX root = ../main.tex

\begin{abstract}[zh]
在IMT-2030定义的6G愿景中,泛在连接与万物智联被确立为下一代信息基础设施的核心目标。低轨卫星网络首次被纳入6G体系架构,凭借低时延、广覆盖与高重访特性,成为支撑全球一体化通信的重要组成部分。随着星上载荷的轻量化与算力的持续提升,低轨星座正加速向具备星上智能推理与协同学习能力的自治系统演进;与此同时,人工智能驱动的网络管控正由地面离线辅助优化逐步过渡到星上在线自组织调度,为实现星座的自感知—自知—自优化运行奠定了技术基础。然而,低轨星座的高速相对运动、星地信道的强动态性、星间链路的频繁切换及业务负载的时空波动共同构成显著的强动态特征;同时,星载资源受限与回程链路的高时延、高丢包又带来弱中心特性,使集中式管控在高动态环境下难以及时响应与全局优化。

在此背景下,本文以端到端服务质量(Quality of Service, QoS)保障为总体目标,构建了一种分布式智能管控体系。该体系以多智能体强化学习(Multi-Agent Reinforcement Learning, MARL)为主要实现手段,将应用层QoS指标——时延、可靠性、连续性与公平性——向下分解为接入层与网络—应用层的多维资源与控制约束,实现从应用意图到物理资源的跨层协同与一致性优化。低轨星座的分布式管控面临多重挑战:一是随机接入体制下海量用户切换易引发信道碰撞与负载集中,导致接入公平性与服务连续性难以兼顾;二是星载资源受限与信道频率选择性衰落并存,致使切片级QoS保证与用户体验不匹配,传统比例公平或轮询调度在异质信道环境下偏离最优;三是网络拓扑时变与星间链频繁切换导致多源丢包与路由失效,而现有多智能体算法在局部观测条件下难以捕获跨节点依赖关系,特征聚合中过压缩与比特级重传低效问题进一步限制了端到端可靠性。针对上述挑战,本文从接入管理、接入网切片优化与端到端语义感知路由三个方面开展研究,主要工作与创新点如下:

(1)在面向公平性保障的海量用户切换管理方面,本文针对低轨星座随机接入架构下公平性与切换效率的权衡问题开展研究,旨在于分布式管控框架下实现系统接入公平性提升与通信连续性保障。为此,本文将公平性优化建模为最大化最小用户接入容量的多目标优化问题,并引入社会福利函数以统一刻画系统吞吐与公平性之间的权衡。构建的MARL框架中,通过邻域消息传播机制实现系统公平性近似估计,并利用特征工程学习多阈值切换判决门限以适应高动态星地信道条件。针对多目标优化的不可分性,推导多目标下策略梯度的解析式,将全局公平性项映射为智能体间通信信息,实现基于邻域交互的低开销协同更新。进一步设计双分支值函数的混合神经网络结构以减少梯度估计偏差,并在分布式异步训练平台上显著提升采样效率。系统级仿真结果表明,该机制在典型低轨星座的随机接入场景下实现了分布式公平接入与卫星负载均衡,有效提升了海量用户环境下的服务连续性与接入效率。

(2)在面向用户级QoS保障的接入网切片优化方面,本文针对星载资源受限与网络动态性下切片级QoS保证与用户体验不匹配的问题,提出一套环境动态感知与自适应的分布式优化框架。该问题被建模为功率与带宽联合分配的双层优化问题,上层由切片调度器(Slice Scheduler)在长时间尺度下执行带宽与功率的粗粒度配置,下层由媒体接入控制调度器(MAC Scheduler)在短时间尺度内完成资源块(Resource Block, RB)级的细粒度分配。上层引入基于生成式世界模型(World Model)的时序生成网络,通过最小化预测与重构损失实现对业务到达率和信道演化的动态建模,从而在时变环境下具备更强的泛化与自适应能力;下层采用交替方向乘子法(Alternating Direction Method of Multipliers, ADMM)构建可并行求解的分布式子问题,实现跨星功率与频谱资源的协同分配。为进一步降低通信与计算开销,本文设计分块式ADMM求解结构,仅需交换局部对偶变量即可逼近全局最优。仿真结果表明,该机制在频谱利用率、QoS保证水平及尾部时延控制方面均显著优于传统集中式或比例公平方法,实现了从环境感知、动态预测到自优化决策的星上边缘自治能力。

(3)在面向端到端可靠保障的语义感知传输与智能路由协同优化方面,本文针对低轨星座中大规模拓扑与星间链频繁切换导致的多源丢包与路径失效问题,提出融合语义通信、图神经网络(Graph Neural Network, GNN)与多智能体强化学习的联合优化体系。在传输层,本文通过任务语义嵌入与重要度建模建立语义特征子空间,实现基于任务相关性的自适应语义压缩与冗余控制,使接收端能够在部分比特丢失的情况下重构关键语义信息,从而克服传统比特级重传在多跳路径下的时延累积。在路由层,构建基于MARL的分布式路径决策机制,以最小时延与链路可靠性最大化为联合优化目标,使卫星节点仅依赖局部观测实现路径选择与拥塞规避。为解决局部观测下的特征提取与信息聚合困难,本文在策略网络中引入GNN感知层以聚合邻域多跳特征,并通过图曲率约束调节信息传播范围与强度,避免特征过压缩并增强拓扑可分辨性。三者协同构成“语义压缩—拓扑感知—智能决策”的自适应闭环,使系统在强动态拓扑下兼顾语义吞吐效率与链路可达性。仿真结果表明,该机制有效缓解了重传冗余与路由更新滞后问题,显著提升了端到端时延性能与语义可靠性,支撑低轨星座在“强动态—弱中心”条件下实现面向QoS的可靠保障。
\end{abstract}

\begin{abstract}[en]
  In the IMT-2030 vision, ubiquitous connectivity and intelligent interconnection of everything have been established as the core goals of next-generation information infrastructure. Low Earth Orbit (LEO) satellite networks are incorporated into the 6G architecture for the first time, becoming a key enabler for global integrated communications owing to their low latency, wide coverage, and high revisit frequency. With the continuous miniaturization of on-board payloads and the enhancement of satellite computing capabilities, LEO constellations are rapidly evolving toward autonomous systems endowed with on-board intelligent reasoning and collaborative learning. Meanwhile, AI-driven network control is shifting from ground-based offline optimization to on-board online self-organized scheduling, laying the foundation for self-awareness, self-cognition, and self-optimization in constellation operations. However, the high relative motion of satellites, strong dynamics of satellite-ground channels, frequent switching of inter-satellite links, and spatio-temporal fluctuations of traffic jointly constitute significant strong dynamics; meanwhile, limited on-board resources, high latency, and high packet loss on the feeder links introduce weak centralization, making centralized control difficult to respond promptly or achieve global optimization in highly dynamic environments.

Against this background, this dissertation aims to achieve end-to-end Quality of Service (QoS) assurance by proposing a distributed intelligent control architecture. Leveraging Multi-Agent Reinforcement Learning (MARL) as the core methodology, the proposed framework decomposes application-layer QoS indicators—latency, reliability, continuity, and fairness—into multi-dimensional constraints across the access and network-application layers, thereby achieving cross-layer coordination and consistency from user intent to physical resources. The distributed control of LEO constellations faces multiple challenges: (1) the random access mechanism leads to massive user handovers, causing channel collisions and load imbalance that hinder both fairness and service continuity; (2) limited satellite resources and frequency-selective fading channels result in inconsistency between slice-level QoS guarantees and user experience, while conventional proportional-fair and round-robin schedulers deviate from the optimum under heterogeneous channel conditions; (3) dynamic topologies and frequent inter-satellite link switching induce multi-source packet loss and routing failures, and existing MARL-based methods struggle to extract effective state representations under local observability, where feature over-compression and inefficient bit-level retransmissions further degrade end-to-end reliability. To address these issues, this work focuses on user access management, access-network slicing optimization, and end-to-end semantic-aware routing, with the main contributions summarized as follows.

(1) In terms of fairness-oriented massive user handover management, this work investigates the tradeoff between fairness and handover efficiency in LEO random access architectures, aiming to enhance access fairness and ensure communication continuity under distributed control. The fairness optimization is formulated as a multi-objective problem that maximizes the minimum user access capacity, with a social welfare function introduced to jointly characterize system throughput and fairness. Within the MARL framework, a neighborhood message-passing mechanism is designed to estimate system fairness, while feature engineering is employed to learn multi-threshold handover decision boundaries under highly dynamic satellite-ground channels. To handle the non-separability of multi-objective optimization, an analytical form of the policy gradient is derived, in which the global fairness term is encoded into inter-agent communications for low-overhead collaborative updates. A dual-branch value network is further designed to reduce gradient estimation bias, and a distributed asynchronous training platform is developed to improve sampling efficiency. System-level simulations demonstrate that the proposed scheme achieves distributed fair access and satellite load balancing in multi-beam LEO random access scenarios, significantly improving service continuity and access efficiency for massive users.

(2) In terms of access-network slicing optimization for user-level QoS assurance, this work addresses the mismatch between slice-level QoS guarantees and actual user experience under resource-constrained and time-varying conditions, and proposes an environment-aware and adaptive distributed optimization framework. The problem is formulated as a two-layer joint optimization of power and bandwidth allocation, where the upper-layer slice scheduler performs coarse-grained allocation over a long time scale, while the lower-layer Medium Access Control (MAC) scheduler executes fine-grained allocation of Resource Blocks (RBs) at a short time scale. The upper layer incorporates a generative world model with a temporal generative network that minimizes prediction-reconstruction loss, enabling dynamic modeling of traffic arrival rates and channel evolution with enhanced generalization and adaptability under dynamic environments. The lower layer employs the Alternating Direction Method of Multipliers (ADMM) to construct distributed subproblems that support parallel solving of multi-satellite power and spectrum allocation, with block-wise ADMM enabling near-global optimality through local dual-variable exchange. Simulation results show that the proposed mechanism significantly improves spectrum efficiency, QoS assurance, and tail latency performance compared to centralized and proportional-fair baselines, achieving on-board edge autonomy from environment perception and dynamic prediction to self-optimized decision-making.

(3) In terms of semantic-aware transmission and intelligent routing for end-to-end reliability assurance, this work addresses multi-source packet loss and path failures caused by dynamic topologies and frequent inter-satellite link switching in large-scale LEO constellations. A joint optimization framework integrating semantic communications, Graph Neural Networks (GNNs), and MARL is proposed. At the transmission layer, task-oriented semantic embedding and importance modeling are introduced to construct a semantic feature subspace that supports adaptive compression and redundancy control based on task relevance, allowing the receiver to reconstruct key semantic information even under partial bit loss—thereby overcoming the inefficiency of bit-level retransmissions over multi-hop routes. At the routing layer, a MARL-based distributed routing decision model is developed with a joint objective of minimizing end-to-end latency and maximizing link reliability, enabling each satellite node to make path-selection and congestion-avoidance decisions using only local observations. To address the challenges of limited observability and over-compressed representations, a GNN perception layer is embedded in the policy network to aggregate multi-hop neighborhood features, and a graph-curvature regularization term is introduced to optimize information propagation intensity and range, mitigating feature over-smoothing and enhancing topological discriminability. Together, these components form an adaptive loop of ``semantic compression-topology perception-intelligent decision-making,'' achieving both semantic throughput efficiency and routing robustness under highly dynamic topologies. Simulation results verify that the proposed mechanism effectively alleviates retransmission redundancy and routing-update latency, significantly improving end-to-end latency and semantic reliability, and enabling QoS-oriented reliable assurance in ``strongly dynamic yet weakly centralized'' LEO constellation environments.
\end{abstract}
