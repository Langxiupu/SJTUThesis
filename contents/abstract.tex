% !TEX root = ../main.tex

\begin{abstract}[zh]
在IMT-2030定义的6G愿景中,泛在连接与万物智联被确立为下一代信息基础设施的核心目标。低轨卫星网络首次被纳入6G体系架构,凭借低时延、广覆盖与高重访特性,成为支撑全球一体化通信的重要组成部分。随着星上载荷的轻量化与算力的持续提升,低轨星座正加速向具备星上智能推理与协同学习能力的自治系统演进;与此同时,人工智能驱动的网络管控正由地面离线辅助优化逐步过渡到星上在线自组织调度,为实现星座的自感知—自知—自优化运行奠定了技术基础。然而,低轨星座的高速相对运动、星地信道的强动态性、星间链路的易失性及业务负载的时空波动共同构成显著的强动态特征;同时,星载资源受限与回程链路的高时延、高丢包又带来弱中心特性,使集中式控制在高动态环境下难以及时响应与全局优化。在此背景下,本文以端到端服务质量保障为总体目标,构建了一种分布式智能管控体系。该体系以多智能体强化学习为核心手段,将应用层QoS(Quality of Service)指标——时延、可靠性、连续性与公平性——向下分解为接入层与网络—应用层的多维资源与控制约束,实现从应用意图到物理资源的跨层协同与一致性优化。在接入层,针对高速相对运动导致的频繁切换与负载失衡,提出公平性导向的用户接入点切换机制,以保障服务覆盖与连接连续性;在跨层资源调度中,针对高动态环境下用户级QoS保障的高维复杂优化问题,提出基于环境动力学建模与预测的跨层调度机制,实现切片与MAC层的自适应协同优化;在网络—应用层,设计融合语义通信与图感知的分布式路由机制,以实现端到端时延与可靠性的联合优化。三者共同构成从应用QoS意图到资源配置的闭环映射与分布式调控,推动低轨星座在“强动态—弱中心”环境下向自治运行与面向QoS的端到端可靠保障的协同演化。本文的主要工作与创新如下:

(1)在面向公平性保障的海量用户切换管理方面,本文针对低轨星座随机接入架构下公平性与切换效率的权衡问题开展研究,旨在于分布式管控框架下提升系统接入公平性并保障通信连续性。由于卫星采用非预约信道的随机接入体制,海量用户在切换过程中易产生信道碰撞,造成接入冲突与资源浪费;同时,地表用户分布的非均匀性使流量过度集中于少数热点卫星,难以充分发挥星座多重覆盖带来的容量潜力。为在分布式场景中实现公平与效率的动态平衡,本文引入多智能体强化学习框架:一方面通过邻域消息传播近似评估系统公平性,另一方面结合特征工程实现多要素切换判决的自适应学习,从而克服传统基于参考信号接收功率判据在高路径损耗环境下的局限。本文将公平性优化建模为最大化最小用户接入容量的多目标优化问题,并通过引入社会福利函数统一刻画系统吞吐与公平性之间的权衡。针对多目标优化的不可分性,推导多目标下多智能体强化学习的策略梯度解析式,并将全局公平性项映射为智能体间的通信信息,实现基于邻域交互的低开销协同更新。进一步,为平衡个体收益与全局公平性,设计双分支值函数的混合神经网络结构以降低梯度估计偏差,并实现分布式异步训练平台以提升采样效率。系统级仿真结果表明,该机制在典型多波束LEO随机接入场景下实现了分布式公平接入与卫星负载均衡,有效提升了海量用户环境下的服务连续性与接入效率。

(2)在面向用户级QoS保障的跨层资源协同优化方面,本文针对星载资源受限与网络动态性下的用户级QoS保障难题,提出一套面向环境动态感知与自适应的分布式优化框架。该问题主要体现在两个方面:其一,接入网切片在资源分配中易出现“资源超卖”,传统依据切片平均性能评估QoS的方式难以在资源受限条件下维持用户级服务一致性,导致切片QoS保证与实际体验脱节;其二,星地信道的频率选择性衰落使不同用户间信道特性差异显著,异质信道环境下传统轮询或比例公平调度难以逼近最优解。针对上述问题,本文设计双层分布式优化框架,包括上层切片调度器与下层媒体接入控制调度器。上层在大时间尺度下执行功率与带宽的粗粒度配置,并基于世界模型对历史业务到达率与资源块级信噪比进行时序建模与生成式预测,以感知业务与信道动态性;下层在小时间尺度下执行细粒度调度,区域内可见卫星基于交替方向乘子法(ADMM)并行地为用户分配功率与时频资源块,并通过星间协调实现全局一致优化。世界模型与ADMM共同构成多粒度资源调度架构:前者赋予系统环境感知与预测能力,后者实现解析化的高效资源映射。仿真结果显示,该机制在提升频谱利用率与QoS保证水平的同时,有效降低尾部时延与QoS违约概率,实现了从环境感知、动态决策到自优化的星上边缘自治能力。

(3)在面向端到端可靠保障的语义感知传输与智能路由协同优化方面,本文针对低轨星座中大规模网络与拓扑时变特性引发的多源丢包与路径失效问题,提出融合语义通信、图曲率优化与多智能体强化学习的联合优化体系。卫星网络的丢包不仅来源于拥塞,也受链路切换与信道衰落影响,使传统依赖比特级重传的机制在多跳环境中效率低下;另一方面,卫星传输数据(如遥感图像、控制指令)蕴含丰富语义信息,适合在语义层进行抽象与压缩表达。基于此,本文在传输层引入语义通信范式,通过任务语义抽象与重要度建模实现语义级压缩与自适应冗余控制,使接收端可基于语义特征重构信息,即使部分比特丢失亦能恢复关键内容。在路由层,针对软件定义网络及开放式最短路径优先协议需全局状态同步、在千星规模下难以扩展的问题,构建基于多智能体强化学习的路由决策机制,使卫星节点仅依赖局部观测实现路径选择与拥塞规避。进一步,考虑卫星网状拓扑下邻居节点数量随跳数指数增长的特性,本文引入图神经网络(Graph Neural Network, GNN)感知层以聚合邻域与多跳特征,并通过图曲率约束重构信息流逻辑拓扑,优化特征传播范围与强度,从而避免特征混叠并提升状态可分辨性。三者协同构成“语义压缩—拓扑感知—智能决策”的自适应闭环,使系统在强动态拓扑下实现语义有效吞吐与链路可达性的统一提升。验证结果表明,该机制有效缓解了重传冗余与路由更新滞后问题,支撑低轨星座在“强动态—弱中心”条件下实现面向QoS的端到端可靠保障。
\end{abstract}

\begin{abstract}[en]
  Shanghai Jiao Tong University (SJTU) is a key university in China. SJTU was
  founded in 1896. It is one of the oldest universities in China. The University
  has nurtured large numbers of outstanding figures include JIANG Zemin, DING
  Guangen, QIAN Xuesen, Wu Wenjun, WANG An, etc.

  SJTU has beautiful campuses, Bao Zhaolong Library, Various laboratories. It
  has been actively involved in international academic exchange programs. It is
  the center of CERNet in east China region, through computer networks, SJTU has
  faster and closer connection with the world.
\end{abstract}
